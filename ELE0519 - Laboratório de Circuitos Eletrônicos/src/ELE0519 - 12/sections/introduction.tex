\section{Introdução}

O presente relatório visa detalhar o experimento laboratorial realizado na disciplina laboratório de circuitos eletrônicos no dia 19 de novembro de 2019 abordando o assunto de filtros ativos, implementando dois filtros Butterworth passa-baixas de 1ª e 2ª ordem, numa configuração que usa   amplificadores operacionais (AMPOPS), mais especificamente os do circuito integrado (CI) TL082. A razão para sua escolha, ao invés do LM741 ou LF353, é o seu maior produto banda-ganho (GBW) que os CIs anteriores.  

A prática possui o objetivo de observar a topologia dos circuitos e determinar experimentalmente a função de transferência e frequência de corte de cada um, permitindo comparar os valores teóricos esperados com os resultados práticos obtidos e também comparar os filtros de 1ª e 2ª ordem entre si.

O primeiro circuito a ser montado e experimentado é um filtro Butterworth passa-baixas de 1ª ordem como pode ser observado no circuito da figura \ref{ckt:1}. O segundo circuito é um filtro Butterworth passa-baixas de 2ª ordem como o=pode ser observado no circuito da figura \ref{ckt:2}.

\begin{figure}[H]
\begin{center}
\tikz \node [scale=0.75, inner sep=0] {
\begin{tikzpicture} [american]
    \draw (0,0) node[op amp, yscale=-1] (opamp) {}
    (opamp.+) -- (-2.5,0.5) to[R, l_=$R$, -o] (-5,0.5)
    (opamp.-) -- (-1.5,-0.5) -- (-1.5,-2) -- (1.5,-2) to[short,-*] (1.5,0)
    (opamp.out) to[short, -o] (2,0) 
    (opamp.down) -- (0,0.5) node[vcc]{-12\,\textnormal{V}}
    (opamp.up) -- (0,-0.5) node[vee]{+12\,\textnormal{V}}
    (2.5,0) node[]{$V_o$}
    (-5.5,0.5) node[]{$V_i$}
    (-2.5,-1.5) node[ground]{} to[C, l=$C$, -*] (-2.5,0.5)
    ;
\end{tikzpicture}
};
\end{center}
\caption{Filtro Butterworth de 1ª ordem ($R=15k\ohm$ e $C=10nF$)}
\label{ckt:1} 
\end{figure}
\begin{figure}[H]
\begin{center}
\tikz \node [scale=0.75, inner sep=0] {
\begin{tikzpicture} [american]
    \draw (0,0) node[op amp, yscale=-1] (opamp) {}
    (opamp.+) -- (-2.5,0.5) to[R, l_=$R_2$, -o] (-5,0.5)
    (opamp.-) -- (-1.5,-0.5) -- (-1.5,-2) -- (1.5,-2) to[short,-*] (1.5,0)
    (opamp.out) to[short, -o] (2,0) 
    (opamp.down) -- (0,0.5) node[vcc]{-12\,\textnormal{V}}
    (opamp.up) -- (0,-0.5) node[vee]{+12\,\textnormal{V}}
    (2.5,0) node[]{$V_o$}
    (-7.5,0.5) node[]{$V_i$}
    (-2.5,-1.5) node[ground]{} to[C, l=$C_2$, -*] (-2.5,0.5)
    (-5,0.5) to[short, *-] (-5,2) to[C, l=$C_1$] (1.5,2) -- (1.5,0)
    (-5,0.5) to[R, l_=$R_1$, -o] (-7,0.5)
    ;
\end{tikzpicture}
};
\end{center}
\caption{Filtro Butterworth de 2ª ordem ($R_1=8.2k\ohm$, $R_2=15k\ohm$, $C_1=22nF$ e $C_2=10nF$)}
\label{ckt:2} 
\end{figure}

O gerador de sinais foi utilizado para gerar os sinais senoidais de entrada na porta $V_i$ de cada circuito. Já o osciloscópio foi utilziado para medir as formas de onda tanto na entrada $V_i$, quanto na saída $V_o$ de cada circuito, permitindo a comparação direta.