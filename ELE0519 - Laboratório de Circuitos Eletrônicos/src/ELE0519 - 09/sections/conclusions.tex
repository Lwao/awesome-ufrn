\section{Conclusões}

A prática permitiu compreender e distinguir os circuitos retificadores de precisão, um categoria de circuitos retificadores utilizando amplificadores operacionais, bem como ajudou a compreender o funcionamento na prática desses circuitos, permitindo entender as limitações impostas.

Podemos notar que os retificadores de precisão conseguem superar o problema imposto pelos circuitos retificadores normais (resistor-diodo), que são problemáticos com
sinais de grandeza menores graças à queda de tensão dos diodos (0.7V para o silício). Também, pode-se perceber que os resultados obtidos na prática estão de acordo com o que foi previsto na análise teórica.

Dessa forma, mesmo com as dificuldades encontradas, permitiu-se encontrar uma ótima correspondência entre os valores teóricos com os práticos, implicando que os circuitos propostos de fato possuem um valor prático de aplicação.

%\newpage

%\section{Anexos}
