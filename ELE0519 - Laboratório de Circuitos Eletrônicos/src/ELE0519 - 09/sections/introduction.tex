\section{Introdução}


O presente relatório visa detalhar o experimento laboratorial realizado na disciplina laboratório de circuitos eletrônicos no dia 22 de outubro de 2019 onde o assunto abordado é sobre retificadores de precisão, que se fazem do uso de amplificadores operacionais (AMPOPS), mais especificamente os do circuito integrado (CI) LM741. 

A prática possui o objetivo de identificar as saídas dos retificadores de meia onda e onda completa de precisão, identificando seus benefícios em contraste com retificadores que usam apenas diodos.

Os circuitos a serem montados podem ser identificados nas figuras \ref{ckt:1} e \ref{ckt:2}, respectivamente o retificador de meia onda de precisão e o retificador de onda completa de precisão.

\begin{figure}[H]
\begin{center}
\tikz \node [scale=0.75, inner sep=0] {
\begin{tikzpicture} [ american, ]
    \draw (0,0) node[op amp] (opamp) {}
    (opamp.+) -- (-1.5,-0.5) node[ground]{}
    (opamp.-) to[R, l_=$R_1$] (-5,0.5) to[sV_=$V_{in}$] (-5,-1.5) node[ground]{}
    (opamp.out) to[short, -*] (1.5,0) 
    (opamp.up) -- (0,0.5) node[vcc]{+12\,\textnormal{V}}
    (opamp.down) -- (0,-0.5) node[vee]{-12\,\textnormal{V}}
    (-1.5,0.5) to[short, *-] (-1.5,4)
    (-1.5,2.5) to[D*, l=1N4148, *-] (1.5,2.5) 
    (1.5,2.5) to[short, -*] (1.5,0)
    (-1.5,4) to[R, l=$R_F$] (1.5,4)
    (1.5,0) to[D*, l=1N4148, -*] (3.5,0) to[short, -*] (4,0) 
    (3.5,0) to[R, l=$R_2$] (3.5,-2) node[ground]{}
    (4.2,0) node[right] {$V_o$}
    (1.5,4) -- (3.5,4) -- (3.5,0)
    
    ;
    
\end{tikzpicture}
};
\end{center}
\caption{Retificador de meia onda. Onde: $R_1=R_F=1.2k \ohm$ e $R_2=12k \ohm$}
\label{ckt:1} 
\end{figure}
\begin{figure}[H]
\begin{center}
\tikz \node [scale=0.75, inner sep=0] {
\begin{tikzpicture} [ american, ]
    \draw (0,0) node[op amp] (opamp) {}
    (opamp.+) -- (-1.5,-0.5) node[ground]{}
    (opamp.-) to[short, -*] (-2,0.5) to[R, l_=$R_1$] (-5,0.5) to[sV_=$V_{in}$] (-5,-1.5) node[ground]{}
    (opamp.out) to[short, -*] (1.5,0) 
    (opamp.up) -- (0,0.5) node[vcc]{+12\,\textnormal{V}}
    (opamp.down) -- (0,-0.5) node[vee]{-12\,\textnormal{V}}
    (-2,2.5) -- (-2,-3) 
    (-2,2.5) to[R, l=$R_2$] (1.5,2.5) to[R, l=$R_4$, *-*] (4,2.5) to[R, l=$R_5$] (7,2.5) 
    (-2,-3) to[R, l=$R_3$] (1.5,-3) 
    (1.5,2.5) to[D*, l=1N4148] (1.5,0) to[D*, l=1N4148, -*] (1.5,-3)
    (1.5,-3.4) node[] {$V_{o1}$}
    (1.5,-3) -- (4,-3) -- (4,-0.5)
    (4,0.5) -- (4,2.5)
    (7,2.5) -- (7,0) to[short, -*] (7.4,0)
    (7.8,0) node[] {$V_{o}$}
    ;
    \draw (5.5,0) node[op amp] (opamp) {}
    (opamp.+) -- (4,-0.5)
    (opamp.-) -- (4,0.5)
    (opamp.out) to[short, -*] (7,0) 
    (opamp.up) -- (5.5,0.5) node[vcc]{+12\,\textnormal{V}}
    (opamp.down) -- (5.5,-0.5) node[vee]{-12\,\textnormal{V}}
    
    
    ;
    
\end{tikzpicture}
};
\end{center}
\caption{Retificador de onda completa. Onde: $R_2=R_3=R_4=R_5=12k \ohm$ e $R_1=6.8k \ohm$}
\label{ckt:2} 
\end{figure}

O gerador de funções foi utilizado para gerar os sinais de entrada para os circuitos, enquanto que o osciloscópio foi utilizado para medir as formas de onda de entrada e saída para serem comparadas.