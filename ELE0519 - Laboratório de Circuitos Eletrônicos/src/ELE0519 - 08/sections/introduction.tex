\section{Introdução}

O presente relatório visa detalhar o experimento laboratorial realizado na disciplina laboratório de circuitos eletrônicos no dia 15 de outubro de 2019 onde o assunto abordado é o circuitos comparadores utilizando amplificadores operacionais (AMPOPs), mais especificamente os do circuito integrado (CI) LM741. Com o objetivo de compreender o funcionamento dos circuitos comparadores simples e com histerese (comparador regenerativo "Schmitt Trigger"). Os esquema dos circuitos analisados são mostradas abaixo.

\begin{figure}[H]
\begin{center}
\tikz \node [scale=0.75, inner sep=0] {
\begin{tikzpicture} [ american, ]
    %shorts
    \draw (0,0) node[op amp,yscale=-1] (opamp) {}
    (opamp.+) to[short,-*] (-1.5,0.5) -- (-2.5,0.5) 
    (opamp.-) -- (-1.5,-0.5)
    (opamp.out) to[short, -*] (5,0)
    (opamp.down) --(0,0.5) node[vcc]{+10\,\textnormal{V}}
    (opamp.up) --(0,-0.5) node[vee]{-10\,\textnormal{V}}
    (-1.5,-0.5) node[ground]{}
    (-3,1.5) to[pR, l_=$15k \ohm$] (-3,-0.5)
    (-3,3.5) to[R, l_=$10k \ohm$, *-] (-3,1.5)
    (-3,-0.5) to[R, l_=$10k \ohm$, -*] (-3,-2.5)
    (-1.5,0.9) node[] {$V_{in}$}
    (2,0) to[R, l=$820 \ohm$, *-] (2,-2) to[leD*] (2,-4) node[ground]{}
    (4,0) to[R, l=$820 \ohm$, *-] (4,-2) 
    (4,-4) to[leD*] (4,-2) 
    (4,-4) node[ground]{}
    (-3.2,-2.5) node[left] {$-10V$} 
    (-3.2,3.5) node[left] {$+10V$}
    (5.1,0) node[right] {$V_{o}$}
    ;
    
\end{tikzpicture}
};
\end{center}
\caption{Comparador simples não inversor.}
\label{ckt:1} 
\end{figure}
\begin{figure}[H]
\begin{center}
\tikz \node [scale=0.75, inner sep=0] {
\begin{tikzpicture} [ american, ]
    %shorts
    \draw (0,0) node[op amp] (opamp) {}
    (opamp.+) -- (-1.5,-0.5)
    (opamp.-) to[short,-*] (-1.5,0.5) -- (-2.5,0.5) 
    (opamp.out) to[short, -*] (6.5,0) 
    (opamp.up) -- (0,0.5) node[vcc]{+10\,\textnormal{V}}
    (opamp.down) -- (0,-0.5) node[vee]{-10\,\textnormal{V}}
    (-1.5,-0.5) -- (-1.5,-2) -- (1.5,-2)    
    (-3,1.5) to[pR, l_=$15k \ohm$] (-3,-0.5)
    (-3,3.5) to[R, l_=$10k \ohm$, *-] (-3,1.5)
    (-3,-0.5) to[R, l_=$10k \ohm$, -*] (-3,-2.5) 
    (-1.5,0.9) node[] {$V_{in}$}
    (3.5,0) to[R, l=$820 \ohm$, *-] (3.5,-2) to[leD*] (3.5,-4) node[ground]{}
    (5.5,0) to[R, l=$820 \ohm$, *-] (5.5,-2) 
    (5.5,-4) to[leD*] (5.5,-2) 
    (5.5,-4) node[ground]{}
    (1.5,0) to[R, l=$12k \ohm$, *-] (1.5,-2)
    (1.5,-2) to[R, l=$1.2k \ohm$, *-] (1.5,-4) node[ground]{}
    (-3.2,-2.5) node[left] {$-10V$} 
    (-3.2,3.5) node[left] {$+10V$}
    (6.6,0) node[right] {$V_{o}$}
    ;
    
\end{tikzpicture}
};
\end{center}
\caption{Comparador regenerativo "Schmitt Trigger".}
\label{ckt:2} 
\end{figure}

Os circuitos comparadores como o próprio nome já diz, permite comparar 2 tensões de entrada, um valor de referência conhecido e outro a ser comparado $V_{in}$, apresentando uma única saída resultado da comparação.

A atividade nos mostra dois circuitos, um comparador não-inversor simples, figura \ref{ckt:1}, que é apenas um AMPOP com tensão de referência no terra na entrada inversora e tensão $V_{in}$ na entrada não-inversora. O outro circuito é claramente comparador inversor com histerese, figura \ref{ckt:2}, pois apresenta realimentação positiva e $V_{in}$ estar na entrada inversora. A partir desses dois circuitos, foi possível efetuar a medição da tensão limiar de entrada com a tensão de referência dada, ver as tensões de saturação de ambos os circuitos, além de principalmente ver a característica de transferência (CT) para o circuito comparador com histerese. 

Além disso, o osciloscópio e o multímetro digital foram utilizados para medir os valores e as formas de onda na entrada e na saída dos comparadores, como forma de comparar os valores e comprovar os resultados teóricos obtidos. Vale destacar que com a medição da faixa de resistência do potenciômetro com o multímetro, verificou que este possuía um valor máximo de resistência de $15k \ohm$ ao invés de $10k \ohm$, assim justificando as mudanças realizadas nos circuitos da figura \ref{ckt:1} e \ref{ckt:2}.