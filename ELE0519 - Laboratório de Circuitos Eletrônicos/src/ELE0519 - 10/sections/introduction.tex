\section{Introdução}




O presente relatório visa detalhar o experimento laboratorial realizado na disciplina laboratório de circuitos eletrônicos no dia 29 de outubro de 2019 no qual o assunto abordado é sobre gerador de onda triangular, que também é capaz de gerar uma onda quadrada em uma de suas saídas, valendo-se do uso de amplificadores operacionais (AMPOPS), mais especificamente os do circuito integrado (CI) LF353. 

A prática possui o objetivo de identificar como as saídas se comportam e interagem entre si, bem como medir os valores de amplitude e frequência das ondas geradas e observar como eles se alteram ao alterar alguns dos componentes do circuito. O circuito a ser montado está apresentado na figura \ref{ckt:1}.

\begin{figure}[H]
\begin{center}
\tikz \node [scale=0.75, inner sep=0] {
\begin{tikzpicture} [ american, ]
    \draw (0,0.5) node[op amp] (opamp) {}
    (opamp.+) -- (-1.5,0) -- (-1.5,-1.5)
    (opamp.-) -- (-2.5,1) -- (-2.5,-0.5) node[ground]{} 
    (opamp.out) to[R, l=$R_4$, -*] (3.5,0.5) to[short, -*] (4.5,0.5) to[R, l=$R_2$] (6.5,0.5)
    (opamp.up) -- (0,1) node[vcc]{+10\,\textnormal{V}}
    (opamp.down) -- (0,0) node[vee]{-10\,\textnormal{V}}
    
    (4.5,1) to[short, *-] (4.5,-1.5) to[zzD*, l_=$2.7 V$] (4.5,0.5)
    (4.5,-1.5) to[zzD*, l=$2.7 V$] (4.5,-3.5) node[ground]{}
    (3.5,0.5) -- (3.5,-1.5) to[R, l=$R_1$, -*] (-1.5,-1.5)
    (4.5,1.5) node[]{$V_{o1}$}
    ;
    \draw (8.5,0) node[op amp] (opamp) {}
    (opamp.+) -- (7,-0.5) node[ground]{}
    (opamp.-) -- (6.5,0.5)
    (opamp.out) to[short, -*] (10,0) to[short, -*] (10.5,0) 
    (11,0) node[]{$V_{o2}$}
    (opamp.up) -- (8.5,0.5) node[vcc]{+10\,\textnormal{V}}
    (opamp.down) -- (8.5,-0.5) node[vee]{-10\,\textnormal{V}}
    (7,0.5) to[short, *-] (7,2.5) to[C, l=$C$] (10,2.5) to[short, -*] (10,0)
    (10,0) -- (10,-5) to[R, l=$R_3$] (-1.5,-5) -- (-1.5,-1.5)
    (0,0.5)  node[]{\Large \textbf{A}}
    (8.5,0)  node[]{\Large \textbf{B}}
    ;
    
\end{tikzpicture}
};
\end{center}
\caption{Gerador de onda quadrada e triangular ($R_1 = 12k \ohm, R_2 = R_3 = 8.2kK \ohm, R_4 = 560 \ohm, C= 22nF$).}
\label{ckt:1} 
\end{figure}

Como circuitos geradores não possuem entradas para sinais externos, foi utilizada apenas a fonte DC para alimentação dos circuitos integrados. O osciloscópio foi utilizado para comparar as formas de onda das saídas $V_{o1}$ e $V_{o2}$.