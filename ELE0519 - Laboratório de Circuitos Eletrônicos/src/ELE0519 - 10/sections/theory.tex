\setlength{\abovedisplayskip}{-25pt}
\setlength{\belowdisplayskip}{-25pt}

\section{Análise Teórica}

Para analisar o circuito da figura \ref{ckt:1} é necessário dividir a análise em duas partes. Primeiro analisando a saída do comparador não-inversor com histerese constituído pelo AMPOP A e depois a saída do circuito integrador ideal formado pelo AMPOP B.

\subsection{Análise do comparador}

A saída $V_{oA}$ do AMPOP A só pode assumir os valores de saturação de $\pm 10 V$. Porém a saída do comparador é obtida em $V_{o1}$, de forma que assumirá valores de acordo com os diodos Zeners ligados a esse nó, de forma que:

\begin{equation} \label{vo1+}
V_{o1} =
\left \{
\begin{array}{cc}
+3.4 V, & V_{oA} = +10 V \\
-3.4 V, & V_{oA} = -10 V \\
\end{array}
\right.
\end{equation}

Assim, para realizar a análise do comparador são necessárias as tensões $V_{A}^{+}$ e $V_{A}^{-}$. Elas podem ser equacionadas em \ref{v+} e \ref{v-}.

\begin{equation} \label{v+}
V_{A}^{+} = V_{o1} + (V_{o2}-V_{o1}) \times \frac{R_1}{R_1 + R_3}
\end{equation}

\begin{equation} \label{v-}
V_{A}^{-} = 0
\end{equation}

Considerando a entrada do comparador como a saída $V_{o2}$ do integrador, realiza-se as seguintes análise para obter a curva de transferência do comparador não-inversor.

\begin{itemize}
    \item Para o caso $V_{o1}=3.4 V$, logo $V_{A}^{+}<V_{A}^{-}$\\
    \begin{center}
        \begin{equation} \label{+>-}
            V_{o2} < 2.32 V
        \end{equation}
    \end{center}
    
    \item Para o caso $V_{o1}=-3.4 V$, logo $V_{A}^{+}<V_{A}^{-}$ \\ 
    
    \begin{center}
        \begin{equation} \label{+<-}
            V_{o2} > -2.32 V
        \end{equation}
    \end{center}
\end{itemize}

Com esses valores, obtém-se a característica de transferência (CT) da figura \ref{graph:1}.


\begin{figure}[H]
\begin{center}
\begin{tikzpicture} 
\begin{axis}[very thick,
                     samples = 100,
                     ytick={-10,10},
                     xlabel = {$V_{o2}[V]$},
                     ylabel = {$V_{o1}[V]$},
                     xmin = -8,
                     xmax = 8,
                     ymin = -5,
                     ymax = 5,
                     axis x line = middle,
                     axis y line = middle,
                     ticks = none]
            \addplot[dashed] plot (\x, 3.4);
            \addplot[dashed] plot (\x,-3.4);
            \addplot[red, name path=A] plot (\x, {-3.4+6.8/(1 + exp(-(\x-2.32)*100))});
            \addplot[red, name path=B] plot (\x, {-3.4+6.8/(1 + exp(-(\x+2.32)*100))});
            \addplot[mark=none] coordinates {(0,3.4)} node[pin=150:{$+3.4V$}]{};
            \addplot[mark=none] coordinates {(0,-3.4)} node[pin=-30:{$-3.4V$}]{};
            \addplot[->] coordinates {(2.32,-0.0001) (2.32,0)};
            \addplot[->] coordinates {(-2.32,0) (-2.32,-0.0001)};
            \addplot[mark=none] coordinates {(-2.32,0)} node[pin=150:{$-2.32V$}]{};
            \addplot[mark=none] coordinates {(2.32,0)} node[pin=-30:{$+2.32V$}]{};
        \end{axis}
        
\end{tikzpicture}
\end{center}
\caption{Característica de transferência para o comparador não-inversor com histerese.}
\label{graph:1} 
\end{figure}


\subsection{Análise do integrador}

O integrador montado com o AMPOP B é ideal e possui uma saída que pode ser encontrada pela equação \ref{int}.

\begin{equation} \label{int}
 V_{o2} = - \frac{1}{R_2 C} \int V_{o1} dt
\end{equation}

Como $V_{o1}$ só pode assumir um valor constante (positivo ou negativo) para certo intervalo de tempo, a equação \ref{int} pode ser desenvolvida fazendo-se a integração, obtendo-se a equação \ref{nint}

\begin{equation} \label{nint}
 V_{o2} = - \frac{V_{o1}}{R_2 C} \times t = m \times t
\end{equation}

O coeficiente angular da função \ref{nint} indicará a inclinação da onda triangular gerada em $V_{o2}$. Esse coeficiente assumirá dois valores, dependendo do valor de $V_{o1}$, assim encontrando a relação obtida em \ref{m}.

\begin{equation} \label{m}
m = -\frac{V_{o1}}{R_2 C} = 
\left \{
\begin{array}{cc}
- 18847 Vs^{-1}, & V_{o1} = +3.4 V \\
18847 Vs^{-1}, & V_{o1} = -3.4 V \\
\end{array}
\right.
\end{equation}

\subsection{Análise completa}

Após obtida a CT do comparador não-inversor com histerese e o coeficiente angular da onda triangular obtida na saída, torna-se possível analisar as formas de onda, sendo uma onda quadrada em $V_{o1}$ e uma triangular em $V_{o2}$, determinando a amplitude de cada uma, bem como a frequência.

A figura \ref{graph:2} mostra as formas de onda com as amplitudes que podem ser obtidas pela CT do comparador. Os limites verticais da CT indica os valores de amplitude que a onda quadrada possui, uma vez que ela é obtida na saída do comparador. Já a amplitude da onda triangular é definida pelos limites horizontais da CT, uma vez que esses valores são controlados pela saída do comparador.


\begin{figure}[H]
\begin{center}
\begin{tikzpicture} 
\begin{axis}[
width=10cm,
height=6cm,
x axis line style={-stealth},
y axis line style={-stealth},
%xticklabels={$t_0$, $t_1$, $t_2$, $t_3$, $ $},
xticklabels={$ $, $ $, $ $, $ $, $ $},
ymax = 4,xmax=5.5, ymin = -4,
axis lines*=center,
ytick={-3.4, -2.32, 0, 2.32, 3.4},
xtick={0, 1, 2, 3, 4},
%xlabel={Time $\rightarrow$},
%ylabel={Amplitude},
xlabel = {$Tempo [s]$},
ylabel = {$Tensão [V]$},]
\addplot+[thick,mark=none,const plot]
coordinates
{(0,-3.4) (1,3.4) (2,-3.4) (3,3.4) (4,-3.4)}; \addlegendentry{$V_{o1}$};
\addplot[thick, red, domain=0:4] coordinates {(0,-2.32)(1,2.32)(2,-2.32)(3,2.32)(4,-2.32)}; \addlegendentry{$V_{o2}$};
\addplot[dashed, domain=0:4] plot (\x, 3.4);
\addplot[dashed, domain=0:4] plot (\x, -3.4);
\addplot[dashed, domain=0:4] plot (\x, 2.32);
\addplot[dashed, domain=0:4] plot (\x, -2.32);

\end{axis}
\draw[<->,thick,black]
    (2.2,-0.4) -- (3.5,-0.4)
    node[midway,above=1pt] {$T_1$};  
    \draw[<->,thick,black]
    (3.5,-0.4) -- (5,-0.4)
    node[midway,above=1pt] {$T_2$};  
\draw (0.5,2) node[] {$t_0$};
\draw (2,2) node[] {$t_1$};
\draw (3.35,2) node[] {$t_2$};
\draw (4.75,2) node[] {$t_3$};
\end{tikzpicture}
\end{center}
\caption{Formas de onda triangular e quadrada obtidas no circuito.}
\label{graph:2} 
\end{figure}


Porém ainda fica a definir a frequência dos sinais, que pode ser facilmente encontrada com base no período da onda triangular e suas inclinações, que podem ser encontradas pela equação \ref{m}.

No primeiro semi-ciclo correspondente a $T_1$ o coeficiente angular é negativo, e no semi-ciclo correspondente a $T_2$ o coeficiente angular é positivo, assim, por meio da equação da reta em \ref{ret} pode-se obter ambos os valores, que somados geram o período total de ambas as ondas e quando invertidos indicam a frequência das ondas geradas.

%\begin{center}
        \begin{equation} \label{ret}
            \frac{\Delta V}{\Delta t} = m
        \end{equation}
%\end{center}

\begin{itemize}
    \item Para $T_1$\\
\end{itemize}

\begin{center}
    $\frac{-2.32 - 2.32}{T_1} = - 18847$
\end{center}

\begin{center}
    \begin{equation} \label{t1}
    T_1 = 0.2461 ms
    \end{equation}
\end{center}

\begin{itemize}
    \item Para $T_2$\\
\end{itemize}

\begin{center}
    $\frac{2.32 - (-2.32)}{T_2} =  18847$
\end{center}

\begin{center}
    \begin{equation} \label{t2}
    T_2 = 0.2461 ms
    \end{equation}
\end{center}


Com os valores obtidos nas equações \ref{t1} e \ref{t2}, pode-se definir a frequência das ondas pela equação \ref{freq}.

\begin{center}
    \begin{equation} \label{freq}
    f = \frac{1}{T_1 + T_2} = 2032.52 Hz
    \end{equation}
\end{center}
