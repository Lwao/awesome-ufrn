\begin{figure}[H]
\begin{center}
\tikz \node [scale=0.70, inner sep=0] {
\begin{tikzpicture} [ american, ]
    \draw (0,0) node[op amp] (opamp) {}
    (opamp.+) -- (-1.5,-0.5) 
    (opamp.-) to[R, l_=$R_1$] (-5,0.5) node[ground]{}
    (opamp.out) to[short, -*] (4,0) 
    (opamp.up) -- (0,0.5) node[vcc]{+10\,\textnormal{V}}
    (opamp.down) -- (0,-0.5) node[vee]{-10\,\textnormal{V}}
    % controle não linear do ganho
    (-1.5,0.5) to[short, *-] (-1.5,5.5)
    (-1.5,3.5) to[R, l=$R_2$, *-*] (1,3.5) 
    (1,4) -- (1,3)
    (1,4) -- (1,3)
    (1,4) to[D*, l=1N4148, -*] (3,4)
    (3,3) to[D*, l=1N4148, *-] (1,3) 
    (-1.5,5.5) to[R, l=$R_3$] (3,5.5)
    (3,5.5) to[short, -*] (3,0)
    (4.2,0) node[right] {$V_o$}
    % realimentação positiva
    (-1.5,-0.5) -- (-1.5,-2.5) to[R, l=$R$, *-] (0.75,-2.5) to[C, l=$C$] (3,-2.5) -- (3,0)
    (-1.5,-2.5) to[short, -*] (-1.5,-3.2)
    (-2.5,-3.2) -- (-0.5,-3.2)
    (-2.5,-3.2) to[R, l=$R$] (-2.5,-5.3)
    (-0.5,-3.2) to[C, l=$C$] (-0.5,-5.3)
    (-2.5,-5.3) -- (-0.5,-5.3)
    (-1.5,-5.3) to[short, -*] (-1.5,-5.3) node[ground]{}
    (5.5,4) node[] {Controle}
    (5.5,3.5) node[] {não linear}
    (5.5,3) node[] {do ganho}
    ;
    \draw[dashed] (1,4) ellipse (3.5cm and 2.3cm)
    ;
\end{tikzpicture}
};
\end{center}
\caption{Oscilador por ponte de Wien. Onde: $R_1=10k\ohm$, $R_2=47k\ohm$, $R_3=22k\ohm$, $R=15k\ohm$ e $C=10nF$}
\label{ckt:1} 
\end{figure}