\begin{figure}[H]
\begin{center}
\begin{tikzpicture} [ american, ]
    \draw (6,-1.75) node[op amp] (opamp3) {}
    (opamp3.+) -- (4,-2.25)
    (opamp3.-) -- (4,-1.25)
    (opamp3.out) --  (7.5,-1.75) node[right] {$V_o$}
    %(opamp2.up) --++(0,0.005) node[vcc]{15\,\textnormal{V}}
    %(opamp2.down) --++(0,-0.005) node[vee]{-15\,\textnormal{V}}
    ;

    \draw (1.5,0) node[left] {$V_2$} to[R,l=$R_1$, o-*] (4,0)
    (1.5,-3.5) node[left] {$V_1$} to[R,l_=$R_{3}$, o-*] (4,-3.5)
    (4,0) -- (4,-1.25)
    (4,-3.5) -- (4,-2.25)
    (4,0) to[R,l=$R_2$] (7,0)
    (4,-3.5) to[R,l_=$R_{4}$] (7,-3.5)
    (7,-3.5) node[ground]{}
    (7,0) to[short,-*] (7,-1.75)
    ;

\end{tikzpicture}

\end{center}
\caption{Amplificador diferencial com AMPOPs.}
\label{ckt:3} 
\end{figure}