\begin{figure}[h!]
\begin{center}
\begin{tikzpicture}
    \draw (2,0) to[R,l=$33\ohm$] (2,2)
    (2,2) to [Tnpn,n=npn2,mirror] (2,3.5)
    (npn2.E) node[left=3mm, above=5mm]{$Q_2$}
    (-2,0) to[R,l_=$33\ohm$] (-2,2)
    (-2,2) to [Tnpn,n=npn1] (-2,3.5)
    (npn1.E) node[right=3mm, above=5mm]{$Q_1$}
    (-2,0) node[ground]{}
    (2,0) node[ground]{}
    (3.2,2.75) node[]{$v_{i2}$}
    (-3.2,2.75) node[]{$v_{i1}$}
    (2,3.5) to[R,l=$4.7\kilo\ohm$] (2,5.5)
    (-2,3.5) to[R,l_=$4.7\kilo\ohm$] (-2,5.5)
    (-3,5.5) node[ground]{}
    (3,5.5) node[ground]{}
    (3,5.5) to[short, -] (2,5.5)
    (-3,5.5) to[short, -] (-2,5.5)
    ;
\end{tikzpicture}

\end{center}
\caption{Circuito equivalente através do teorema do semi-circuito.}
\end{figure}