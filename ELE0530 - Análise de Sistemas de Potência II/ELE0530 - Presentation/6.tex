\section{Fatores que influenciam a estabilidade transitória}
\begin{frame}
\frametitle{Fatores que influenciam a estabilidade transitória}
\begin{itemize}
\item Carregamento do gerador;
\item Local da falta e seu tipo;
\item Tempo de extinção de falta;
\item Reatância do gerador (baixa reatância aumenta o pico de potência e reduz o ângulo inicial do rotor);
\item Inércia do gerador (maior inércia, menor taxa de variação do ângulo, logo menor energia cinética ganha durante a falta - $A_1^{'}<A_1$);
\item Tensão interna e externa (barramento infinito) do gerador.
\end{itemize}
\end{frame}

\begin{frame}
\begin{table}[h]
\centering
\caption{Tabela.}\label{tb:data}
\begin{tabularx}{\linewidth}{cccc}
Edifício & Tipo de climatização & FP & Carregamento (\%) \\ \hline
\multicolumn{1}{|c|} Instituto Internacional de Física & VRF & 0.944 & 12 \\
\multicolumn{1}{|c|} Escola de Ciências e Tecnologias & Expansão direta & 0.936 & 58 \\ \hline
\end{tabularx}
\end{table}

\end{frame}