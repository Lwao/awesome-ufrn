\begin{subquestion}
    \item De acordo com o gráfico da figura \ref{graph:1} e considerando a função degrau unitário como $u(t)$, pode-se determinar a representação matemática do sinal $s(t)$ como:
    
    \begin{center}
        $s(t) = \frac{A}{2}[u(t) - 2u(t-T/2) - u(t-T)]$
    \end{center}
    \begin{figure}[H]
\centering
\tikz \node [scale=0.8, inner sep=0] {
\begin{tikzpicture} 
    \draw[black, very thick] (-2,0) -- (0,0);
    \draw[black, very thick] (0,0) -- (0,3);
    \draw[black, very thick] (0,3) -- (2,3);
    \node[blue] at (-2.4,0) {$R_{in}$};  
    \node[red] at (2.4,3) {$R_{L}$};  
    \node[black] at (1,1.5) {$1+Q_p^2$};  
    \draw[>=latex, <->] (0.2,0) -- (0.2,3);
    \node[black] at (0,-0.5) {(a)};
\end{tikzpicture}
};
\hspace{1cm}
\tikz \node [scale=0.8, inner sep=0] {
\begin{tikzpicture} [american]
    \draw[>=triangle 90, ->] (-0.5,0) -- (0.5,0);
    \draw[>=triangle 90, ->] (6,0) -- (7,0);
    \node[black] at (-0.9,0) {$R_{in}$}; 
    \node[black] at (5.6,0) {$R_L$}; 
    \node[black] at (3,-4) {(b)};
    \draw (1,0) to[C, l=$C$, *-*] (5,0)
    (4,0) to[L, l=$L$, *-] (4,-3) node[ground]{}
    ;
\end{tikzpicture}
};

\caption{(a) Desired effect of impedance gain (b) L matching network for a parallel to serial transformation of load impedance. Source: own.}
\label{graph:1} 
\end{figure}
    
    Como a resposta ao impulso para o filtro casado é dado por: $h(t) = s(T-t)$, tem-se que a representação gráfica é dada pela figura \ref{graph:2} e a sua representação matemática será:
    
    \begin{center}
        $h(t) = \frac{A}{2}[u(T-t) - 2u(T/2-t)] - u(-t)]$
    \end{center}
    
    \begin{figure}[H]
\begin{center}
\begin{tikzpicture} 
\tikzset{every pin edge/.style={scale=0.00001}    }
\begin{axis}[very thick,
                     samples = 100,
                     ytick={-0.25,2},
                     xlabel = {$t$},
                     ylabel = {$y(t)$},
                     xmin = -1,
                     xmax = 7,
                     ymin = -0.25,
                     ymax = 1.5,
                     axis x line = middle,
                     axis y line = middle,
                     ticks = none]
                     
            % waveform
            \addplot[mark=none] coordinates {(0+2,0) (1+2,1)};
            \addplot[mark=none] coordinates {(1+2,1) (3+2,0)};
            
            \addplot[mark=none] coordinates {(0+1,0) (1+1,1-0.5)};
            \addplot[mark=none] coordinates {(1+1,1-0.5) (3+1,0)};
            
            \addplot[mark=none] coordinates {(0+3,0) (1+3,1-0.5)};
            \addplot[mark=none] coordinates {(1+3,1-0.5) (3+3,0)};
            
            % labels
            \addplot[mark=none] coordinates {(1,0.2)} node[pin=270:{\scriptsize$t_d-T$}]{};
            \addplot[mark=none] coordinates {(2,0.2)} node[pin=270:{\scriptsize$t_d$}]{};
            \addplot[mark=none] coordinates {(3,0.2)} node[pin=270:{\scriptsize$t_d+T$}]{};
            \addplot[mark=none] coordinates {(0.8,1)} node[pin=180:{$A$}]{};
            \addplot[mark=none] coordinates {(0.9,0.5)} node[pin=180:{$A/2$}]{};
            
            % dashed line
            \addplot[dashed] coordinates {(0,1) (3,1)};
            \addplot[dashed] coordinates {(0,0.5) (4,0.5)};
            
            
        \end{axis}
\end{tikzpicture}
\end{center}
\caption{Forma de onda do sinal $y(t)$, como uma composição de sinais $g(t)$, com : $t_d>T$.}
\label{graph:2} 
\end{figure}
    
    \item Enquanto isso a saída do filtro casado é dada pela convolução entre o sinal de entrada e o filtro, de forma que: $r(t) = s(t) \ast h(t)$. Apesar da tarefa ser exaustiva, a convolução será calculada por partes levando em consideração a representação gráfica. Sendo assim:
    
    \begin{center}
        $0 \leq t \leq T/2$: $r(t) = -\frac{A^2}{4}t$ \\ \vspace{2pt}
        $T/2 < t \leq T$: $r(t) = -\frac{A^2T}{8} + \frac{3A^2}{4}(t-T/2)$ \\ \vspace{2pt}
        $T < t \leq 3T/2$: $r(t) = \frac{A^2T}{4} - \frac{3A^2}{4}(t-T)$ \\ \vspace{1pt}
        $3T/2 < t \leq 2T$: $r(t) = -\frac{A^2T}{8} + \frac{A^2}{4}(t-3T/2)$ \\ \vspace{1pt}
    \end{center}
    
    
\begin{figure}[H]
\centering
\tikz \node [scale=0.8, inner sep=0] {
\begin{tikzpicture} 
    \draw[black, very thick] (-2,1) -- (0,1);
    \draw[black, very thick] (0,1) -- (0,3);
    \draw[black, very thick] (0,3) -- (2,3);
    \draw[black, very thick] (2,3) -- (2,-1);
    \draw[black, very thick] (2,-1) -- (4,-1);
    \node[blue] at (-2.4,1) {$R_{in}$};  
    \node[red] at (4.4,-1) {$R_{L}$}; 
    \node[black] at (1,3.4) {$R_{i}$}; 
    \node[black] at (-1.2,2) {$1+Q_2^2$}; 
    \node[black] at (3.2,1) {$1+Q_1^2$}; 
    \draw[>=latex, <->] (-0.2,1) -- (-0.2,3);
    \draw[>=latex, <->] (2.2,-1) -- (2.2,3);
    \node[black] at (0,-1.5) {(a)};
\end{tikzpicture}
};
\hspace{1cm}
\tikz \node [scale=0.8, inner sep=0] {
\begin{tikzpicture} [american]
    \draw[>=triangle 90, ->] (-2.5,0) -- (-1.5,0);
    \draw[>=triangle 90, ->] (6.5,0) -- (5.5,0);
    \draw[] (2,0.3) -- (2,0.8);
    \draw[>=triangle 90, ->] (2,0.3) -- (2.5,0.3);
    \node[black] at (2,1.2) {$R_{i}$}; 
    \node[black] at (-2.9,0) {$R_{in}$}; 
    \node[black] at (6.9,0) {$R_L$}; 
    \node[black] at (2,-4) {(b)};
    \draw (2,0) to[L, l=$L_1$, *-*] (5,0)
    (2,0) to[C, l=$C$, *-] (2,-3) node[ground]{}
    (-1,0) to[L, l=$L_2$, *-*] (2,0)
    ;
\end{tikzpicture}
};

\caption{(a) Desired effect of impedance gain(b) T matching network for a serial to parallel transformation of load impedance. Source: own.}
\label{graph:3} 
\end{figure}
    
    \item O valor máximo pode ser facilmente encontrado pelo gráfico \ref{graph:3} ou pela expressão matemática como $\frac{A^2T}{4}$ em $t=T$, ou seja, com pico em um instante de amostragem igual ao tempo de símbolo.
\end{subquestion}

\newpage