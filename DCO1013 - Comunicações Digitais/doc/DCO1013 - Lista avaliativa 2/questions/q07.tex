Para um sinal NRZ a probabilidade média do erro é conhecida como:

\begin{center}
    $P_e = Q\left(\sqrt{\frac{2E_b}{N_0}}\right)$
\end{center}

Como a energia de bit é: $E_b = A^2T_b$; e a variância do ruído: $\sigma^2 = \frac{N_0}{2T_b} = 10^{-2} V^2$, tem-se:

\begin{center}
    $P_e = Q\left(\sqrt{\frac{A^2T_b}{\sigma^2T_b}}\right) = Q\left(\frac{A}{\sigma}\right)$
\end{center}

Como é desejado que haja $1$ bit errado para cada $10^8$ bits transmitidos, faz-se a probabilidade do erro igual a $P_e=10^{-8}$. O desvio padrão é a raiz quadrada da variância, logo $\sigma = 10^{-1} V$. Fazendo $u = A/\sigma$, e considerando a função \textit{qfuncinv()} do MATLAB/Octave:

\begin{equation} \label{q07:1}
    \frac{A}{\sigma} = qfuncinv(P_e) = 5.6120
\end{equation}

Resolvendo $A/\sigma = 5.6120$, implica em $A = 0.5612$.









\begin{comment}
Para um sinal NRZ a probabilidade média do erro é conhecida como:

\begin{center}
    $P_e = \frac{1}{2} erfc\left(\sqrt{\frac{E_b}{N_0}}\right)$
\end{center}

Como a energia de bit é: $E_b = A^2T_b$; e a variância do ruído: $\sigma^2 = \frac{N_0}{2T_b} = 10^{-2} V^2$, tem-se:

\begin{center}
    $P_e = \frac{1}{2} erfc\left(\sqrt{\frac{A^2T_b}{2\sigma^2T_b}}\right) = \frac{1}{2} erfc\left(\frac{A}{2\sigma^2}\right)$
\end{center}

Como é desejado que haja $1$ bit errado para cada $10^8$ bits transmitidos, faz-se a probabilidade do erro igual a $P_e=10^{-8}$. O desvio padrão é a raiz quadrada da variância, logo $\sigma = 10^{-1} V$. Fazendo $u = A/\sigma$, tem-se que:

\begin{equation} \label{q07:1}
    P_e = \frac{1}{2} erfc(u) = \frac{1}{\sqrt{\pi}} \int_{u}^{\infty} e^{-z^2} dz
\end{equation}

Como a probabilidade do erro é pequena, pode-se tentar aproximar o resultado a:

\begin{center}
    $P_e = \frac{e^{-u^2}}{2\sqrt{\pi}u}$
\end{center}

Resolvendo iterativamente para $u = A/(2\sigma) = 3.97189$, logo $A = 0.794378$.
\end{comment}





