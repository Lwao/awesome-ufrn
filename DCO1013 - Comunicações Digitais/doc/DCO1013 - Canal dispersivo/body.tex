% Set Title, Author, and email
\title{DCO1013 - canal dispersivo}
\author{Levy Gabriel da S. G. \\ Engenharia elétrica - UFRN}

\maketitle
\thispagestyle{fancy}

Segundo as considerações impostas pelo exercício, a saída y(t) do canal será:

\begin{center}
    $y(t) = g(t) \ast h(t)$
\end{center}

No domínio tem-se:

\begin{center}
    $Y(f) = G(f) \ast H(f)$
\end{center}

Onde $G(f)\rightleftharpoons g(t)$ representa o sinal sujeito ao canal, $H(f)$ o canal e $Y(f)$ a saída do canal. Assim, desenvolvendo a equação baseado na definição do canal:

\begin{center}
    $Y(f) = G(f)e^{-j2\pi f t_d} + k \, cos(2\pi fT) G(f)e^{-j2\pi f t_d}$
\end{center}

De acordo com a fórmula de Euler, o cosseno pode ser expresso como uma soma de exponenciais no domínio em questão (no caso da frequência):

\begin{center}
    $cos(2\pi f T) = \frac{1}{2} (e^{j2\pi f T}   +    e^{-j2\pi f T})$
\end{center}

Incorporando essa expressão à equação da saída $Y(f)$:

\begin{center}
    $Y(f) = G(f)e^{-j2\pi f t_d} + \frac{k}{2} \, G(f)   [e^{-j2\pi f (t_d-T)}   +    e^{-j2\pi f (t_d+T)}]$
\end{center}

Considerando a propriedade de deslocamento no tempo da transformada de Fourier para um deslocamento genérico $x$: $G(f)e^{-j2\pi f x} \rightleftharpoons g(t-x)$. Com isso pode-se determinar a expressão do tempo na saída por meio dessa propriedade considerando valores de atraso para $x$ como $t_d$, $t_d-T$ e $t-d+T$, assim a expressão para a saída será:

\begin{center}
    $y(t) = g(t-t_d) + \frac{k}{2} [g(t-t_d+T) + g(t-t_d-T)]$
\end{center}

Outra forma de enxergar essa expressão é fazer $t=t+t_d$:

\begin{center}
    $y(t+t_d) = g(t) + \frac{k}{2} [g(t+T) + g(t-T)]$
\end{center}

Isso permite observar que a saída é a ponderação acima baseada na função $g(t)$ atrasada de $t_d$.

Para ilustrar a forma de onda, não será considerada a forma de onda genérica de $g(t)$ como aquela apresentada no problema, mas sim uma forma modificada para facilitar a representação gráfica, como a da figura \ref{graph:1}.

\begin{figure}[H]
\centering
\tikz \node [scale=0.8, inner sep=0] {
\begin{tikzpicture} 
    \draw[black, very thick] (-2,0) -- (0,0);
    \draw[black, very thick] (0,0) -- (0,3);
    \draw[black, very thick] (0,3) -- (2,3);
    \node[blue] at (-2.4,0) {$R_{in}$};  
    \node[red] at (2.4,3) {$R_{L}$};  
    \node[black] at (1,1.5) {$1+Q_p^2$};  
    \draw[>=latex, <->] (0.2,0) -- (0.2,3);
    \node[black] at (0,-0.5) {(a)};
\end{tikzpicture}
};
\hspace{1cm}
\tikz \node [scale=0.8, inner sep=0] {
\begin{tikzpicture} [american]
    \draw[>=triangle 90, ->] (-0.5,0) -- (0.5,0);
    \draw[>=triangle 90, ->] (6,0) -- (7,0);
    \node[black] at (-0.9,0) {$R_{in}$}; 
    \node[black] at (5.6,0) {$R_L$}; 
    \node[black] at (3,-4) {(b)};
    \draw (1,0) to[C, l=$C$, *-*] (5,0)
    (4,0) to[L, l=$L$, *-] (4,-3) node[ground]{}
    ;
\end{tikzpicture}
};

\caption{(a) Desired effect of impedance gain (b) L matching network for a parallel to serial transformation of load impedance. Source: own.}
\label{graph:1} 
\end{figure}

Assim, a forma de onda da saída será a seguinte composição:

\begin{figure}[H]
\begin{center}
\begin{tikzpicture} 
\tikzset{every pin edge/.style={scale=0.00001}    }
\begin{axis}[very thick,
                     samples = 100,
                     ytick={-0.25,2},
                     xlabel = {$t$},
                     ylabel = {$y(t)$},
                     xmin = -1,
                     xmax = 7,
                     ymin = -0.25,
                     ymax = 1.5,
                     axis x line = middle,
                     axis y line = middle,
                     ticks = none]
                     
            % waveform
            \addplot[mark=none] coordinates {(0+2,0) (1+2,1)};
            \addplot[mark=none] coordinates {(1+2,1) (3+2,0)};
            
            \addplot[mark=none] coordinates {(0+1,0) (1+1,1-0.5)};
            \addplot[mark=none] coordinates {(1+1,1-0.5) (3+1,0)};
            
            \addplot[mark=none] coordinates {(0+3,0) (1+3,1-0.5)};
            \addplot[mark=none] coordinates {(1+3,1-0.5) (3+3,0)};
            
            % labels
            \addplot[mark=none] coordinates {(1,0.2)} node[pin=270:{\scriptsize$t_d-T$}]{};
            \addplot[mark=none] coordinates {(2,0.2)} node[pin=270:{\scriptsize$t_d$}]{};
            \addplot[mark=none] coordinates {(3,0.2)} node[pin=270:{\scriptsize$t_d+T$}]{};
            \addplot[mark=none] coordinates {(0.8,1)} node[pin=180:{$A$}]{};
            \addplot[mark=none] coordinates {(0.9,0.5)} node[pin=180:{$A/2$}]{};
            
            % dashed line
            \addplot[dashed] coordinates {(0,1) (3,1)};
            \addplot[dashed] coordinates {(0,0.5) (4,0.5)};
            
            
        \end{axis}
\end{tikzpicture}
\end{center}
\caption{Forma de onda do sinal $y(t)$, como uma composição de sinais $g(t)$, com : $t_d>T$.}
\label{graph:2} 
\end{figure}



