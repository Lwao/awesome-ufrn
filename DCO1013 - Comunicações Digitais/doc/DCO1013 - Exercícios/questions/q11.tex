\textbf{INSS 2014 - 57}

Como as modulações M-QAM e M-PSK modulam a fase da portadora (no caso do QAM, também modula na amplitude), estas não envolvem diretamente a frequência e ocupação de banda. Ou seja, esquemas com M>4 promovem uma maior eficiência espectral, pois ainda ocupam a mesma frequência.

A distância entre os pontos da constelação e o aumento na probabilidade de erro de bit são grandezas inversamente proporcionais, pois uma maior distância entre os pontos permite que haja menor ambiguidade na decisão entre símbolos.

A modulação M-QAM é uma forma melhorada do M-PSK, pois enquanto que os símbolos do M-PSK estão contidos em uma única circunferência, os símbolos do M-QAM estão dispersos ao longo do diagrama de constelação em circunferências de raios distintos ou blocos retangulares que possuem distanciamento diferente do símbolo ao centro do diagrama. Essa organização do M-QAM permite uma menor probabilidade de erro de bit, para uma mesma eficiência espectral, pois a variação da amplitude dos símbolos proporciona um grau de liebrdade a mais.