\textbf{Petrobrás 2010 - 4}

Como em banda passante a largura de banda dobra em relação à banda base, a faixa disponível que antes era 4kHz torna-se 2kHz. A ocupação de banda mínima em banda base para o caso binário é 4.8kHz, porém esta largura não compreende dentro dos 2kHz em banda base disponíveis, justificando que esquemas binários não podem ser atendidos nesse problema (e isso não considera a forma de pulso como \textit{raised cosine}). Dessa forma a largura de banda que se encaixa para o problema é 1.6kHz derivada de M=8 ($log_2(8)=3$). 

A escolha se reduz aos esquemas de modulação 8-PSK e 8-FSK. Assim, o 8-PSK pode ser utilizado justificando sua menor ocupação de banda que o 8-FSK.