\textbf{Comperve 2012 - 26}

Sinais modulados digitalmente podem tanto ser dada com formas de onda descontínuas, como contínuas no tempo. A exemplificar, no BPSK a informação de 0's e 1' é codificada como -1's e 1's de forma que geram descontinuidades de $\pi$ na fase da portadora, enquanto que em esquemas de CPFSK buscam transições mais suaves entre os símbolos para permitir uma menor largura de banda do sinal modulado.

Um dos motivos que justifica o uso da demodulação não coerente no FSK é o fato de que este esquema de modulação pode ser escrito como dois ASK's somados e devidamente ponderados.

Como o conteúdo de informação do PSK reside em sua fase, obrigatoriamente sua demodulação deve ser coerente, pois desvios de fase nos instantes de amostragem podem comprometer totalmente a detecção dos símbolos.

O QAM não exatamente é mais vantajoso que as demais modulações em um canal AWGN, pois este ainda sofre dos efeitos deste tipo de ruído, pois sua informação está tanto em amplitude e fase. 