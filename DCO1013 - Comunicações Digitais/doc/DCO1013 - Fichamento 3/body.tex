% Set Title, Author, and email
\title{DCO1013 - Detecção coerente e não coerente}
\author{Levy Gabriel da S. G. \\ Engenharia elétrica - UFRN}

\maketitle
\thispagestyle{fancy}


Os princípios da detecção coerente e não coerente dependem se a portadora na recepção está em fase ou não com a portadora da transmissão. No caso da detecção coerente a portadora gerada localmente está em fase com portadora da transmissão, de forma que essa configuração permite um melhor desempenho na detecção, porém exige um receptor mais complexo e com altos requisitos técnicos. Em contrapartida a portadora local na detecção não coerente não está em fase com a portadora do transmissor, de forma que o receptor se torna mais simples e de fácil implementação e integração, porém seu desempenho é inferior à coerente. 

A seguir será mostrado matematicamente a diferença entre um sinal tratado em um receptor coerente que promove a recuperação da portadora para a demodulação contra um receptor não coerente que não se preocupa com a recuperação da portadora.

\subsubsection*{Sheng Chen. Notas de Aula. Coherent and Non-coherent Receivers. University of Southampton, School of Electronics and Computer Science. Disponível em: <http://www.ecs.soton.ac.uk/∼qc/EL334/>.}

Considerando o sinal recebido como: $s(t) = A cos(\omega_c t + \varphi) + n(t)$; onde $\omega_c$ é a velocidade angular da portadora e $n(t)$ representa o AWGN. Agora considera-se que a portadora local é: $cos(\omega_c t+ \Bar{\varphi})$, para cada caso ter-se-á (considera-se que :

\begin{itemize}
    \item Recuperação de portadora para demodulação:
    
    Na detecção coerente existe um circuito para recuperação da fase da portadora, como por exemplo um circuito \textit{phase lock loop} (PLL), resultando na seguinte relação entre as fases iniciais da portadora:
    
    \begin{equation}
        \Delta \varphi = \varphi - \Bar{\varphi} \to 0 \, \, \, \, \, \, assim, \, \Bar{\varphi} \to \varphi
    \end{equation}

    Aplicado o circuito de recuperação de fase no sinal recebido, o sinal banda base demodulado, com $x(t)$ representando o sinal banda base transmitido, será:
    
    \begin{equation}
        y(t) = x(t) + n(t)
    \end{equation}
    
    \item Não recuperação de portadora para demodulação:
    
    Na detecção não coerente não haverá um circuito para que alinhe a fase das portadoras, podendo ser aplicada algum algoritmo de estimação da fase da portadora, resultando na seguinte relação entre as fases iniciais da portadora:
    
    \begin{equation}
        \phi = \Delta \varphi = \varphi - \Bar{\varphi} \neq 0 \, \, \, \, \, \, assim, \, \Bar{\varphi} \neq \varphi
    \end{equation}

    Agora o sinal banda base demodulado, com $x(t)$ representando o sinal banda base transmitido, será corrompido por um desvio de fase que prejudica a amostragem:
    
    \begin{equation}
        y(t) = x(t)e^{j\phi} + n(t)
    \end{equation}
\end{itemize}



\subsubsection*{Lathi, B. P., Ding, Z., "Modern Digital and Analog Communication Systems", 4a edição, Editora Oxford University Press, 2009.}



\paragraph{Detecção coerente}

Durante a detecção coerente, para alcançar o mesmo padrão de probabilidade de erro, o ASK deve ter 3dB a mais na potência do PSK para que possuam o mesmo desempenho. Dito isso, em termos de detecção coerente, prefere-se usar o PSK ao invés de ASK, pois uma vez que será investido em complexidade do receptor, que ao menos economize a metade da potência.

Para os esquemas de modulação ASK e PSK, a detecção coerente pode ser implementada de duas formas junto ao filtro casado: uma considera um filtro casado ajustado à forma de pulso em RF, recebendo diretamente o sinal com a portadora; a outra, que é equivalente à anterior, considera a demodulação coerente do sinal RF primeiramente multiplicando pela portadora, de forma que o produto resultante em banda base é encaminhado para o filtro casado.

\paragraph{Detecção não coerente}

Na detecção coerente o PSK é mais preferível que o ASK. Porém, como a detecção não coerente não pode ser aplicada no PSK, pois a informação reside na fase (e esta não possui sincronismo na detecção não coerente), o ASK se torna um esquema de modulação útil nesse cenário (e.g. comunicações óticas).

No caso da demodulação em um detector não coerente e modulação ASK, esta pode ser realizada por meio de um filtro casado para a RF desconsiderando a fase, porém com amplitude máxima divergindo do instante de amostragem. Apesar disso a amplitude estará perto do máximo no instante de amostragem e a saída do filtro casado pode ser tratada por um detector de envoltória.

No caso do FSK deverá haver filtro casados para os pulsos de RF, detectores de envoltória, amostragem e dispositivo comparador para tomada de decisão.

Do ponto de vista prático, o FSK é mais preferível que o ASK, pois: FSK possui limiar ótimo fixo enquanto que este limiar  no ASK depende da energia do sinal; como FSK modula em frequência, este é menos suscetível a desvanecimento que o ASK que possui informação na amplitude; como o FSK realiza a comparação entre os conteúdos em diferentes frequências, estes estarão igualmente degradados pelo canal, assim não prejudicando a detecção.

Porém junto aos benefícios do FSK sobre o ASK, também existe uma desvantagem, que é a maior largura de banda que o FSK ocupa. 

Anteriormente foi dito que o PSK só permite detecção coerente, porém existe um esquema PSK, chamado DPSK, que permite que seja aplicada a detecção não coerente. Isso, pois a informação modulada no DPSK não mais carrega o conteúdo de fase, mas sim sua variação.

