\begin{subquestion}
    \item A distorção de um sinal é caracterizada pela deformação da sua forma de onda ou a relação entre suas componentes frequenciais, causando usualmente a degradação do sinal. Uma amplificação ou atenuação só realizam distorção se atuaram desigualmente na amplitude do sinal, em diferentes componentes frequenciais ou que alterem a relação de fase entre componentes harmônicas de uma forma de onda.
    \item Uma distorção linear atua com a resposta de um sistema linear no sinal, enquanto que no caso da distorção não linear, a modelagem de uma função resposta é impossibilitada, sendo necessário recorrer a outras abordagens (estatística).
\end{subquestion}