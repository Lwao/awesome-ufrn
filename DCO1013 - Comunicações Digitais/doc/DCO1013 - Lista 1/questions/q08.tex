No início do estudo de amostragem do sinal analógico, foi imposta a condição de que o
sinal a ser amostrado, embora arbitrário, deve possuir energia finita. Qual a razão para
esta imposição?

De acordo com a relação de Parseval a energia do sinal pode ser vista do ponto de vista do sinal no tempo e do seu espectro de frequência:

\begin{equation}
    E_g = \int_{-\infty}^{\infty} |g(t)|^2dt = \int_{-\infty}^{\infty} |G(f)|^2df 
\end{equation}

Assim, a razão para que o sinal a ser amostrado seja limitado em energia é devido a relação acima. Isso pois caso o sinal possua energia infinita, a densidade espectral de energia também será infinita, ou seja, o espectro se estende infinitamente ao longo do eixo das frequências, não sendo possível ser repetido periodicamente sem que ocorra mascaramento.