Técnicas de modulação são essenciais para a transmissão de um sinal, pois caso este seja transmitido assim como gerado, estará mais sujeito aos efeitos nocivos do canal ou não possuirão as características necessárias para a transmissão. Assim a modulação permite que a forma do sinal a ser transmitido seja adequada às formas de onda do canal. No caso da modulação em banda base, os sinais transmitidos são pulsos modulados e para a modulação em banda passante a mensagem modula uma portadora senoidal de maior frequência que a banda do base do sinal, assim permitindo que esses sinais sejam transmitidos por rádio frequência, uma vez que as dimensões da antena depende do comprimento de onda da onda a ser transmitida.

A modulação também pode ser utilizada para que vários sinais sejam transmitidos ocupando o mesmo canal, seja por divisão no tempo, divisão na frequência, etc. Ela também permite a diminuição de interferências. Por fim, ela também pode ser utilizada para adaptar a banda do sinal para que atenda alguma necessidade de projeto como filtragem e amplificação.