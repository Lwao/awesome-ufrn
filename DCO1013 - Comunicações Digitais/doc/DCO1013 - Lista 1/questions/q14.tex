\begin{subquestion}
    \item Na modulação por largura de pulso (PWM) a informação é codificada pela duração em que o sinal PWM permanece ligado, ou seja, larguras maiores implica em um sinal de amplitude maior, e vice-versa. Porém, a variação na largura gasta muita energia, desta forma a modulação por posição de pulso (PPM) atua solucionando esse problema. O sinal PPM pode ser gerado a partir do sinal PWM e ele consiste em pulsos estreitos ao final de cada pulso PWM, desta forma sendo energeticamente mais eficiente, mais imune ao ruído, porém deve ter largura ponderada, pois de acordo com que diminui a largura para reduzir a potência, a banda se estende.
    \item As modulações pulsadas PAM, PWM e PPM são analógicas, pois elas são proporcionais à amplitude do sinal, no caso do PAM essa proporcionalidade se dá pela amplitude do pulso, para o PWM na largura do pulso e para o PPM na posição do pulso. O que as diferencia das modulações por ondas contínuas é o fato de utilizarem pulsos para transportar a informação, de forma que consomem potência por curtos períodos de tempo (princialmente no caso do PPM) enquanto que as modulações por onda contínua consomem potência constantemente.
\end{subquestion}
