Um sinal analógico é aquele capaz de assumir infinitos valores, enquanto que um sinal digital só pode assumir um conjunto finito de valores. Enquanto que em sistemas de comunicação digital os sinais trabalhados são, essencialmente, digitais, após a emissão do sinal modulado no canal, este não será mais digital, mas sim analógico. Isso se deve, pois no canal esses sinais terão dimensões físicas (sinal eletromagnético), de forma que possuirão continuidade ao longo da sua amplitude e que, fisicamente, não pode ser quebrada.