Considerando que um trem de pulsos periódico de período $T_0$ pode ser representado pela série de Fourier, tem-se:

\begin{equation}
    \delta_{T_0}(t) = \frac{1}{T_0} \sum_{n = -\infty}^{\infty} e^{jn\omega_0t}
\end{equation}

Assim a representação do sinal amostrado $g(t)$ da questão anterior pode ser expresso como:

\begin{equation}
     g_{T_0}(t) = \frac{1}{T_0} \sum_{n = -\infty}^{\infty} g(t)e^{jn\omega_0t}
\end{equation}

Ao aplicar a transformada de Fourier no sinal amostrado, observa-se que se trata de um somatório de um sinal que possui translação em frequência. Sendo a transformada de Fourier do sinal $g(t)$ é $G(f)$, tem-se que:

\begin{equation}
     G_{T_0}(f) = \frac{1}{T_0} \sum_{n = -\infty}^{\infty} G(f-\frac{n}{T_0})
\end{equation}

Sendo $f_0=T_0$ e considerando um trem de impulsos deslocados na frequência para representar o espectro periódico do sinal amostrado, finalmente tem:

\begin{equation}
     G_{T_0}(f) = f_0 \sum_{n = -\infty}^{\infty} G(nf_0)\delta(f-nf_0) 
\end{equation}
