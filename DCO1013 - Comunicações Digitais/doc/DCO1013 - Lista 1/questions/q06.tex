Considerando um sinal $g(t)$ contínuo no tempo e um trem de impulsos $\delta_T(t)$ periódico e com período $T$, tem-se que o sinal $g(t)$ amostrado em um período $T$ será:

\begin{equation}
    g_T(t) = g(t)\delta_T(t)
\end{equation}

Como o trem de impulsos é periódico, ele pode ser representado por meio de uma série de deltas de Kronecker deslocadas em períodos inteiros de $T$:

\begin{equation}
    \delta_T(t) = \sum_{n = -\infty}^{\infty} \delta(t-nT)
\end{equation}

Substituindo na equação do sinal amostrado, tem-se:

\begin{equation}
     g_T(t) = \sum_{n = -\infty}^{\infty} g(t)\delta(t-nT)
\end{equation}

Da definição da delta de Kronecker, tal que:

\begin{equation} 
\centering
\delta(t) =
\left \{
\begin{array}{cc}
1, & t=0 \\
0, & t\neq0
\end{array}
\right.
\end{equation}

Observa-se que a função só possui valor diferente de zero na origem, assim ao considerar impulsos atrasados de $nT$, estes só assumirão valor nesse instante de tempo. Assim, quando o sinal $g(t)$ for multiplicado pelo trem de impulsos, ele fará com que seus valores sejam aqueles referentes ao instante de tempo $nT$, $\forall \, n \in \mathbb{Z} $, tal que:

\begin{equation}
     g_T(t) = \sum_{n = -\infty}^{\infty} g(t-nT)
\end{equation}