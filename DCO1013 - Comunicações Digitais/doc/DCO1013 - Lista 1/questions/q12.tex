\begin{subquestion}
    \item Amplificadores tradicionais além de amplificar o sinal, eles também adicionam ruído ao sinal amplificado. Para exemplificar, considera-se um sinal que está a 40 dB acima do nível do ruído e variando de -100 a -60 dBW; o sinal na saída desse amplificador está a 30 dB acima do nível de ruído, mas ele varia de -70 a -40 dBW. Essa menor excursão da relação sinal-ruído se dá pelo ruído adicionado pelo amplificador com figura de ruído de 10 dB. Neste contexto introduz-se os amplificadores de baixo ruído (LNA), que se possuem uma figura de ruído relativamente baixa. Os LNAs são essenciais para a comunicação, pois durante a transmissão, o sinal pode ser amplificado em diversos pontos até chegar ao seu destino, de forma que múltiplos estágios de amplificação com LNAs contribuem para manter a SNR do sinal alto.
    \item A direcionalidade da antena contribui para o aumento do ganho, pois uma vez que os lóbulos laterais de irradiação são reduzidos, a onda propagada evita d ese chocar com o solo que é uma grande fonte de ruídos, consequentemente reduzindo a figura de ruído.
\end{subquestion}