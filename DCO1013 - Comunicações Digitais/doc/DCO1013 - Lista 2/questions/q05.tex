Dados: $B=3 \, kHz$, $f_s = 1.33\times2B$, $\Delta_{max}/2=0.5\% \, m_p$ 

A frequência de amostragem será $f_s = 8 \, kHz$ de acordo com a banda fornecida. No que diz respeito ao erro de quantização, este é dado pela razão entre a excursão máxima do sinal e a quantidade de níveis de quantização:
\begin{equation}
	\Delta_{max} = \frac{2m_p}{L} 
\end{equation}
Porém como o erro de quantização máximo é dado como uma função da amplitude máxima do sinal, pode-se encontrar a quantidade de níveis de quantização como:
\begin{equation}
	L = \frac{m_p}{\Delta_{max}/2} = \frac{m_p}{0.5\times0.01\times m_p} = 200 
\end{equation}
Encontrada a quantidade de níveis, arredonda-se para o próximo valor na base de 2 que é 256 níveis e equivale a $n=8$ bits. Assim a largura de banda mínima requerida para transmissão será:
\begin{equation}
	C_{1s} = nf_s = 64 \, kbps 
\end{equation}
E para 24 sinais semelhantes:
\begin{equation}
	C_{24s} = 24\times C_{1s} = 1536 \, kbps 
\end{equation}

Assim, como pode-se transmitir até 2 bits/s por hertz de largura de banda, a largura de
banda mínima de transmissão para ambos os casos será:
\begin{equation}
	B_{1s} = \frac{C_{1s}}{2} = 32 \, kHz 
\end{equation}
\begin{equation}
	B_{24s} = \frac{C_{24s}}{2} = 768 \, kHz 
\end{equation}