Dados: $B=4 \, kHz$, $f_s=2B$, $\mu = 100$, $L_1 = 64$, $L_2 = 256$, $n_1 = 6$, $n_2 = 8$

A SNR de saída para um compandor de lei $\mu$ é dada por:
\begin{equation}
	SNR = \frac{3L^2}{[ln(1+\mu)]^2} 
\end{equation}
Assim, para cada solução a SNR será: $SNR_1 = 27.61 \, dB$, $SNR_2= 39.65 \, dB$. Por meio dos resultados observa-se um incremento de 12 dB de acordo com o aumento de níveis (linearmente significa um aumento de 4 vezes).

Para o caso da banda mínima, pelas equações da questão anterior a capacidade será: $C_1 = 48\, kbps$, $C_2 = 64\, kbps$. Em termos de banda mínima, ter-se-á:  $B_1 = 24\, kHz$, $B_2 = 32\, kHz$. Isso demonstra um aumento de 33\% na banda mínima ocupada de uma solução para outra. 

Isso mostra que com a quantização não-uniforme, a SNR foi quadruplicada com uma relativo baixo aumento de banda.
