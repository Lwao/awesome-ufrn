\begin{subquestion}
    \item A primeira forma de conversão analógica/digital estudada é o PCM. As etapas do processo de digitalização para o PCM possuem o objetivo de transformar a amplitude de um sinal analógico em um sinal digital por meio de bits que codificam a mensagem. A primeira etapa é a amostragem que recorta pulsos da amplitude do sinal analógico em determinado intervalo de tempo entre pulsos e de duração de pulsos. Vale lembrar que a duração do pulso deve ser menor do que o período de amostragem. A segunda etapa é a quantização que vai decidir os valores digitais que se correspondem com faixas de amplitude do sinal. Por fim tem a codificação, que vai atribuir um código binário para as amplitudes definidas.
    \item Enquanto que a SNR do PPM é proporcional ao quadrado da banda, enquanto que no PCM a SNR aumenta exponencialmente com o aumento do número de bits (diretamente relacionado com a banda).
    \item O PCM tem por desvantagem necessitar de vários bits para transmitir a mensagem e possuir uma grande largura de banda. Outro fato é que na quantização, por depender da amplitude do sinal, sendo esta suficiente grande para que ultrapasse a faixa de valores do quantizador, o erro de quantização gerado será alto.
\end{subquestion}