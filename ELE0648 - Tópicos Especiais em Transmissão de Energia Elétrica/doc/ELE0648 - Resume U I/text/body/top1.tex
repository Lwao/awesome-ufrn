\section{Introdução}


\begin{comment}
Apesar da definição generalizada, elas são caracterizadas de diversas formas. Este documento tratará apenas das sobretensões temporárias. O documento também tratará de um resumo básico da análise do regime permanente para que seja possível entender as sobretensões.
\end{comment}


As sobretensões são definidas como qualquer tensão que supere o pico da senoide referente à tensão nominal da linha de transmissão. Um dos tipos de sobretensões são as sobretensões temporárias, comumente conhecidas como sobretensões sustentadas e possuem duração de diversos ciclos e com baixo amortecimento. Elas ocorrem devido uma manobra ou falta, que faz com que uma sobretensão com frente de onda lenta seja gerada. Essa sobretensão só pode ser sanada quando os dispositivos de proteção atuarem.

Desta forma, sua importância na coordenação e isolamento se dá no fato de que equipamentos sujeitos a esse tipo de sobretensão devem ser capazes de suportá-la por muito tempo.

\begin{comment}
A solução da rede em regime permanente permite obter as sobretensões temporárias de origens já discutidas, inclusive diante de fenômenos lineares, quando a parcela transitória está amortecida. Há casos em que as sobretensões de longa duração apresentem constantes de tempo não amortecidas, assim inviabilizam a análise em regime permanente. 

Porém, no caso não linear, com a presença de circuitos chaveados ou transformadores em saturação, recorre-se à simulação no tempo das sobretensões. Como as ondas distorcidas contêm harmônicas em seu espectro, a rede elétrica deve estar bem representada nas frequências naturais de oscilação, de forma a garantir a confiabilidade na resposta no tempo em termos numéricos.
\end{comment}

As causas das sobretensões temporárias são diversas, entre elas:

\begin{itemize}
    \item Energização de linhas;
    \item Rejeição de cargas;
    \item Curto-circuito;
    \item Abertura de fase em linhas de transmissão;
    \item Religamento monopolar;
    \item Polo preso.
\end{itemize}






%\lstinputlisting[language=Matlab,basicstyle=\small]{simplest_test_results.m}

\newpage