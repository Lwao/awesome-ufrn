\section{Polo preso - Religamento Monopolar}

Para o modelo descrito na figura \ref{fig:1}, as reatância e capacitâncias podem ser tratadas em termos de impedâncias, sendo a impedância da parte superior $Z$ e a da parte inferior (que conecta com a terra) $Z_0$. Sendo:

\begin{equation} \label{slide2:z1}
    Z = -\frac{j}{\omega(C_1-C_0)} // j\frac{X_0X_1}{X_0-X_1}
\end{equation}
\begin{equation} \label{slide2:z0}
    Z = -\frac{j}{\omega C_0} // jX_0
\end{equation}

Para uma linha com a fase $a$ desconectada, a tensão e impedância de Thévenin serão: $E_t = -E/2$ e $Z_t = Z/2$. Já a tensão na fase aberta é:

\begin{equation} \label{slide2:va}
    V_a = \frac{E}{2}\frac{Z_0}{1.5Z+Z_0}
\end{equation}

Essa tensão é encontrada pelo circuito simplificado série da tensão $E_t$ e das impedâncias $Z_t$, $Z_0$ e $Z$. Ao aterrar o ponto $V_a$, a corrente que circula será a de arco secundário.

A partir da equação \ref{slide2:va} observa-se que há um valor de compensação reativa que pode gerar uma tensão infinita, esse valor de ressonância será tal que: $Z_0 = -1.5Z$. Assim, desenvolvendo $Z_0$ e $Z_1$, com $X_0=kX_1$, pode-se encontrar que a relação entre a reatância e as capacitâncias será:

 \begin{equation} \label{slide2:x1}
    X_1 = \frac{k+0.5}{k\omega (C_1 +0.5C_0)}
\end{equation}

Com o valor de reatância de ressonância encontrado, valores como a potência reativa pode ser encontrada: $Q_1 = V^2/X_1 \, [volt^2/\Omega] \, M[var]$. 

Uma impedância $Z$ muito alta representa uma anulação da tensão e corrente de arco secundário. Isso significa um desacoplamento das fases da linha de transmissão. A viabilização do religamento monopolar é feito com condições que favoreçam a extinção dessa corrente. Para isso são projetados reatores de neutro para aterramento que possibilite a extinção, ao máximo, dessa corrente. A corrente de arco secundário pode ser calculada como:

 \begin{equation} \label{slide2:is}
    I_S = \frac{V}{3Z}
\end{equation}

Assim, igualando $Z$ a zero, permite-se encontrar a seguinte condição de ressonância que desacopla as fases, resultando em:

 \begin{equation} \label{slide2:xn}
    X_n = \frac{X_1^2\omega (C_1-C_0)}{3[1-X_1\omega (C_1-C_0)]}
\end{equation}

\subsection{Aspectos de estudo}

\begin{itemize}
    \item O arco secundário será extinto se as tensões nas fases abertas não forem elevadas;
    \item Condições próximas da ressonância devem ser evitadas devido a imprecisão dos valores;
    \item O arco secundário pode ser extinto no religamento monopolar  com valores de corrente e tensões de restabelecimento adequados.      
\end{itemize}
