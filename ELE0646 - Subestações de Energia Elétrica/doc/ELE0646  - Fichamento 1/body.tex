% Set Title, Author, and email
\title{ELE0646 - Introdução a subestações de energia elétrica}
\author{Levy Gabriel da S. G. \\ Engenharia elétrica - UFRN}

\maketitle
\thispagestyle{fancy}

\textbf{Classificação de subestações}
\begin{itemize}
    \item Função:
    \begin{itemize} 
        \item SE de manobra ou seccionamento;
        \item SE de transformação
        \begin{itemize} 
            \item SE elevadora: subestação central de transmissão (SCT);
            \item SE abaixadora: subestação receptora de transmissão (SRT).
        \end{itemize}
        \item SE de distribuição;
        \item SE de regulação de tensão;
        \item SE conversora.
    \end{itemize}
    \item Modo de instalação:
    \begin{itemize}
        \item SE externa ou ao tempo;
        \item SE interna ou abrigada;
        \item SE blindada;
    \end{itemize}
    \item Nível de tensão:
    \begin{itemize}
        \item SE de alta tensão ($V_N<230kV$);
        \item SE de extra alta tensão ($V_N>230kV$).
    \end{itemize}
    \item Forma de operação:
    \begin{itemize}
        \item SE com operador;
        \item SE semi-automática;
        \item SE automatizadas;
    \end{itemize}
\end{itemize}

\textbf{Fornecimento de energia (SE do consumidor}

\begin{itemize}
    \item $75kW < C_{INST} < 300kW$: SE ao tempo e $V_1=13.8kV$;
   	\begin{itemize}
    	\item TRAFO ancorado em poste ($150kVA$);
    	\item TRAFO em bancada suportado por dois postes ($225kVA$ e $300kVA$).
	\end{itemize}
    \item $300kW < C_{INST} < 2.500kW$: SE abrigada com postos de medição, proteção e transformação e $V_1=13.8kV$;
    \item $C_{INST} > 2.500kW$: SE ao tempo e $V_1=69kV$;
\end{itemize}

\textbf{Subestação do consumidor}
\begin{itemize}
    \item Entrada de serviço;
    \item Ponto de derivação;
    \item Ponto de entrega;
    \item Ramal de entrada;
    \item Ramal de ligação;
\end{itemize}