\subsection{Aterramento}


Todas as partes metálicas não energizadas da subestação abrigada (portas, janelas, telas de proteção, ferragens, tanques de equipamentos, etc.) devem ser aterradas e ligadas ao sistema de aterramento com cabo de cobre nu de seção mínima de 50 mm² conforme dimensionamento mecânico da NBR 15751.

A malha de terra será interligada por condutores de cobre nu de seção mínima de $50 \,  \text{mm}^2$, conforme NBR 15751, enterrados a uma profundidade mínima de $60 \, \text{cm}$ e devidamente soldados por solda exotérmica entre seus pontos de contato. A dimensão total da malha de terra será um pouco maior que as dimensões da subestação, para que atenda corretamente toda a estrutura e imediações, portanto esta deverá ter 10m por 6m medidos a partir do centro da subestação. Os condutores de ligação da malha de terra deverão ter um comprimento de 1m a partir de cada conexão.

As medições realizadas concluem que a resistência de aterramento é compatível com o Quadro 4 da norma. Para uma corrente de curto-circuito fase-terra maior que 600 A, a resistência de aterramento se mantém abaixo de $20 \Omega$.

O desenho referente à malha de aterramento está constante no apêndice de desenhos estruturais, figura \ref{malha_terra}.
