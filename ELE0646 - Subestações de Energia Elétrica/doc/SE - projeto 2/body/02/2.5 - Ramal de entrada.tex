\subsection{Ramal de entrada}



Considerando a tabela 12 da norma DIS-NOR-036, para uma tensão primária de 13,8 kV e demanda máxima estabelecida no projeto de $510$ kVA, os condutores do ramal de entrada subterrâneo serão de cabo de cobre de seção mínima de $50 \, \text{mm}^2$ para cabos tipo EPR com isolação para 12/20 kV, unipolares, temperatura ambiente de $30^{\circ}$ e eletroduto DN de 100 (4").

A instalação e materiais do ramal de entrada são de inteira responsabilidade do consumidor e este é responsável pela conservação de seus componentes. O ramal de entrada subterrâneo deve partir de um poste particular instalado no interior da propriedade do cliente. A interligação dos cabos nus do poste do ponto de entrega devem ser conectados por meio de mufla com a rede do ramal de entrada subterrâneo.

Na base do poste da mufla e a no máximo 30 m da base devem existir poços subterrâneos do tipo PP, com dimensões de 1,2 m x 0,8 m x 1,3 m (comprimento, largura e profundidade). Deve ser prevista uma volta de cabo com 15 vezes o diâmetro do cabo nos poços de transição da rede aérea para subterrânea para emergências futuras.

A autorização para ocupação do poste da rede aérea para derivação do ramal subterrâneo fica a critério da Distribuidora, que analisará a
solicitação contendo as justificativas técnicas.

O eletroduto externo de descida junto ao poste de derivação deve ser de aço-carbono zincado pelo processo de imersão a quente, dimensionado conforme a Tabela 13 da norma DIS-NOR-03, com altura mínima de 6 m acima do solo e ser fixado ao poste de forma adequada com cintas ajustáveis, arame de aço galvanizado 12 BWG ou bandagens. O eletroduto deve ser vedado na extremidade para evitar a entrada de água.

Não é permitida a instalação do ramal subterrânea em poste que tenha instalado qualquer tipo de equipamento (transformador, religador, chave a óleo, etc.).

%Quando houver cerca metálica sob o ramal, a mesma deve ser seccionada e devidamente aterrada conforme a norma ABNT NBR 15688. 

O ramal de entrada aéreo deve obedecer aos afastamentos mínimos em relação às paredes das edificações, sacadas, janelas, escadas, terraços ou locais assemelhadas definidos pelas normas ABNT NBR 15688 e ABNT NBR 15992.

O apêndice Desenhos estruturais possui a planta detalha da situação do poste do ponto de entrega e da ligação do ramal de entrada para a subestação.


