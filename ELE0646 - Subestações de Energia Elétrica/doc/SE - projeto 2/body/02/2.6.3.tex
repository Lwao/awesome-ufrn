\subsubsection{Outras dimensões da subestação}

\subsubsubsection{Altura da subestação}

A metodologia para se determinar a altura da subestação é baseada na norma NBR 14039, extraída do livro do Mamede \cite{mamede}. A altura da subestação ($H_{se}$) é composta pela seguinte soma:

\begin{equation}
    H_{se} = H_1 + H_2 + H_3 + H_4 + H_5
\end{equation}

Onde: 
$H_1$ - altura total do transformador;
$H_2$ - afastamento da chave seccionadora;
$H_3$ - altura da chave seccionadora;
$H_4$ - altura da isolador;
$H_5$ - afastamento do barramento;

A altura total do transformador é de $1490mm$, com base no maior transformador (225 kVA) obtida ao consultar o catálogo do fabricante. Afastamento da chave seccionadora de $300mm$. Altura da chave seccionadora de $443mm$. Altura do isolador de $150mm$. Por fim, de acordo com a tabela 12.3 do livro do Mamede \cite{mamede}, para uma instalação de 13.8 kV e NBI de 95 kV, a distância mínima fase-terra e fase-fase, ou seja, o afastamento do barramento deve ser mínimo de $160mm$. Contabilizando tudo, a altura da subestação obtida é de:

\begin{equation}
    H_{se} = 1490mm \;+\; 300mm \;+\; 443mm \;+\; 150mm \;+\; 60mm = 2783mm  
\end{equation}

\subsubsubsection{Dimensões internas da subestação}

Considerando que o cubículo reserva possui mesmas dimensões que o cubículo do transformador de 225 kVA, o comprimento total da subestação será:

Comprimento: 

\begin{equation}
    L_{se} = 2000mm \;+\; 1700mm \;+\; 1760mm \;+\; 1725mm \;+\; 1725mm = 8910mm
\end{equation}

Largura (maior valor do cubículo do transformador $= 2270mm$; locais de manobra de acordo com a tabela 12.1 e figura 12.26 do Mamede \cite{mamede} $= 1200mm$; e profundidade média de um Quadro Geral de Força (QGF) $= 900mm$):

\begin{equation}
    C_{se} = 2270mm \;+\; 1200mm \;+\; 900mm  = 4370mm
\end{equation}

\subsubsubsection{Porta de acesso principal}

As subestações devem ser providas de portas metálicas ou inteiramente revestidas de chapas metálicas, com dispositivo antipânico com largura mínima dependente da maior dimensão do maior transformador de acordo com a equação abaixo:

\begin{equation}
    L_p = D_t \;+\; 600mm = 1270mm \;+\; 600mm = 1870mm
\end{equation}

A altura da porta será de 2.10 m. Todas as portas devem abrir para fora.

\subsubsubsection{Janelas de ventilação}

O dimensionamento das janelas de ventilação segue a tabela 16 da norma DIS-NOR-036. A potência escolhida para o dimensionamento foi a potência total de transformação, ou seja 525 kVA. Assim, serão escolhidas 5 janelas cuja área mínima livre por janela seja de 15000 cm³ e dimensões mínimas de 4cm x 100cm x 75cm. 

Todas as telas metálica de proteção deverão ser de malha de 10mm com arame número 12 BWG. 