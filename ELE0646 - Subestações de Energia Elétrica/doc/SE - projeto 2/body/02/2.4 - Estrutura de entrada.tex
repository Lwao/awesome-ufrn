\subsection{Estrutura de entrada}


Considerando a tabela 12 da norma DIS-NOR-036, para uma tensão primária de 13,8 kV e demanda máxima estabelecida no projeto de $510$ kVA de acordo com o memorial de cálculo em anexo, os condutores do ramal de ligação e de entrada subterrâneo serão de cabo de cobre de seção mínima de $50 \, \text{mm}^2$ e eletroduto DN de 100 (4").

O ramal de ligação derivará de um ramal alimentador pertencente à rede da Cosern de média tensão de distribuição e será de inteira responsabilidade da empresa distribuidora. Este, a princípio, deve ser aéreo%, mas podendo ser subterrâneo por necessidade técnica
. O ramal de ligação deve entrar pela frente do terreno e não pode cruzar terreno de terceiros ou passar sob áreas construídas e %, quando aéreo, 
deve estar livre de obstáculos e visível em toda a sua extensão. O ramal de ligação também deverá possuir tamanho máximo de $40 m$.

O ramal de ligação aéreo é indicado nos desenhos estruturais no apêndice.

%Em consequência do estabelecimento da distância máxima do ponto de entrega, o ramal de ligação também deverá possuir comprimento máximo correspondente à distância máxima do ponto de entrega.