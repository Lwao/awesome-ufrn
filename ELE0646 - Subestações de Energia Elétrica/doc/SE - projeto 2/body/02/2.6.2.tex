\subsubsection{Cubículo de medição}

Os equipamentos do cubículo de medição são dimensionados, fornecidos e instalados sob a responsabilidade da concessionária de energia elétrica. Após a ligação dos equipamentos destinados à medição, estes devem ficar inacessíveis aos consumidores (há um selo ou lacre no medidor). O acesso ao medidor somente pode ser feito, a qualquer tempo, pelos colaboradores da concessionária.

A medição deve ser feita no circuito primário de média tensão por meio de 3 x transformadores de corrente (TC) e 3 x transformadores de potencial (TP). Estes devem ser instalados em cavalete metálico, firmemente fixado com parafusos, de acordo com a Figura 32 do ANEXO III da norma DIS-NOR-036. Este cavalete metálico também deve seguir as dimensões propostas na Figura 32 do ANEXO III. A distância entre os transformadores de medição e a caixa de medição deve ser de, no máximo, 10 m.

\begin{comment}
As especificações dos TCs podem se basear nos seguintes dados do fabricante:

% http://www.soltran.com.br/transformadores.asp
% http://www.soltran.com.br/pdf/SN1.pdf
% http://www.soltran.com.br/pdf/SN12.pdf

\begin{itemize}
    \item Fabricante: Soltran Transformadores;
    \item Modelo: SN1;
    \item Número de série: 30170056859;
    \item Tensão máxima de operação: 15 kV;
    \item Frequência: 60 Hz;
    \item Tensão suportável nominal à frequência industrial durante 1 minuto (eficaz): 34 kV;
    \item NBI: 95 kV;
    \item Relação de transformação: 15;
    \item Corrente primária nominal: 75;
    \item Corrente secundária nominal: 5 A;
    \item Exatidão para proteção: 0.3C25;
    \item Fator térmico: 1.2;
    \item Corrente térmica nominal: $I_T = 80 \times I_{N,P} = 80 \times 75 = 6kA$;
    \item Corrente dinâmica nominal: $I_D = 2.5 \times I_{T,P} = 2.5 \times 6000 = 15kA$ ;
    \item Meio Isolante: sólido (epóxi).
\end{itemize}

As especificações dos TPs podem se basear nos seguintes dados de fabricante:

\begin{itemize}
    \item Fabricante: Soltran Transformadores;
    \item Modelo: SN12;
    \item Número de série: 30121034010040;
    \item Classe de tensão: 15 kV;
    \item Frequência: 60 Hz;
    \item Tensão suportável nominal à frequência industrial durante 1 minuto (eficaz): 34 kV;
    \item NBI: 95 kV;
    \item Tensão primária nominal: 13,8/√3 kV;
    \item Tensão secundária nominal: 115;
    \item Relação nominal:  69.28 ;
    \item Grupo de ligação: 2;
    \item Exatidão para proteção (classe e carga): 0.3P75;
    \item Potência térmica nominal: 500 VA;
    \item Meio Isolante: sólido (epóxi).
\end{itemize}
\end{comment}

O cubículo de medição deve possuir um extintor de gás carbônico (CO2) na parte externa, na área de circulação interna e junto à porta de acesso, além de garantir os critérios mínimos da Norma Regulamentadora NR 23 - Proteção Contra Incêndios. A porta de acesso ao cubículo de medição deve possuir dobradiças com abertura somente para o lado externo e ter dispositivo para instalação de selo ou lacre pela empresa distribuidora.

O cubículo de medição deve ocupar um espaço de $1.6 \times 2.0m$

\subsubsection{Cubículo de disjunção}

O disjuntor tripolar de média tensão para uso interno deve estar de acordo com a ABNT NBR IEC 62271-100, com as seguintes especificações:

% https://www.adsdisjuntores.com.br/produto/disjuntor-a-vacuo-modelo-d27-u-ff-off-board-com-carrinho-630a/

\begin{itemize}
    \item Fabricante: ADS disjuntores;
    \item Modelo: D27-U-FF - Off Board com Carrinho 630 A;
    \item Tensão máxima de operação: 15 kV;
    \item Corrente nominal: 630 A;
    \item Frequência: 60 Hz;
    \item Sistema de interrupção a vácuo;
    \item Dispositivo de abertura manual e automática (bobina de abertura);
    \item Capacidade de interrupção sob curto-circuito: 20 kA;
    \item Tensão suportável nominal à frequência industrial durante 1 minuto (eficaz): 34 kV;
    \item NBI: 95 kV;
    \item Sem religamento automático, salvo casos especiais sob consulta à Distribuidora;
    \item Tempo de abertura: 35 ms;
    \item Tempo de fechamento: 50 ms;
    \item Tempo de interrupção: 95 ms;
    \item Comando motorizado: 220 Vca;
    \item Bobina de Trip: 220 Vca;
\end{itemize}

O cubículo de proteção deve ter as seguintes dimensões:

\begin{equation}
    D_{cp} = D_d + 1000mm
\end{equation}

Onde: $D_{cp}$ é a dimensão do cubículo: comprimento ($L$) ou largura ($C$), em mm; e $D_d$ é a dimensão do disjuntor referida à direção em que se quer medir a dimensão do cubículo, em mm.

De acordo com o Mamede \cite{mamede}, "de modo geral, os disjuntores do tipo aberto, da classe 15 kV, 600 A, do tipo aberto, e capacidade de ruptura de até 500 MVA, têm comprimento frontal de aproximadamente 700 mm e uma profundidade de 900 mm." Dito isso, este memorial seguirá essas orientações, resultando em um comprimento do cubículo de $L = 700mm \;+\; 1000mm = 1700mm$ e largura de $C = 900mm \;+\; 1000mm = 1900mm$

\subsubsection{Cubículos de transformação}

Como equipamentos de transformação de força foram escolhidos dois transformadores para atender a demanda máxima de $510 \, \text{kVA}$ e ambos com mesma impedância para possuírem igual carregamento. Os transformadores devem estar abrigados em cubículos exclusivos para cada transformador. Sua isolação será a óleo mineral e refrigeração exclusiva a ar natural (AN), classe de 15 kV, primário em delta e secundário em estrela aterrada.

Os transformadores devem ser devidamente ensaiados e com duas vias do laudo entregues à distribuidora. Sendo estes transformadores a óleo, os laudos devem estar de acordo com a exigências minímas propostas pela norma DIS-NOR-036.

Segue as especificações dos dois transformadores a serem utilizados e que estão de acordo com a tabela 5 da norma DIS-NOR-036:

\subsubsubsection{Transformador 1}
\begin{itemize}

%https://www.weg.net/catalog/weg/BR/pt/Gera%C3%A7%C3%A3o%2C-Transmiss%C3%A3o-e-Distribui%C3%A7%C3%A3o/Transformadores-a-Seco/Pequeno-%28At%C3%A9-300kVA%29/Transformador-Seco-300-0kVA-13-8-0-22kV-CST-IP-00-AN/p/14543066
    \item Fabricante: WEG;
    \item Número de série: 2018899046451;
    \item Potência nominal: 300 kVA;
    \item Tensão primária: 13,8 kV;
    \item Tensão secundária: 380 V;
    \item Taps: -4 x 0.6kV;
    \item Data de fabricação: 27/01/2021;
    \item Corrente de excitação: 2,0\%;
    \item Perdas em vazio: 1300 W;
    \item Perdas em cargas: 4500 W;
    \item Perdas totais: 5800 W;
    \item Impedância: 0,055 pu;
    \item Refrigeração: AN;
\end{itemize}

\subsubsubsection{Transformador 2}
\begin{itemize}

%https://www.weg.net/catalog/weg/BR/pt/Gera%C3%A7%C3%A3o%2C-Transmiss%C3%A3o-e-Distribui%C3%A7%C3%A3o/Transformadores-a-Seco/Pequeno-%28At%C3%A9-300kVA%29/Transformador-Seco-225-0kVA-13-8-0-38kV-CST-IP-00-AN/p/14908883
    \item Fabricante: WEG;
    \item Número de série: 2018731328330;
    \item Potência nominal: 225 kVA;
    \item Tensão primária: 13,8 kV;
    \item Tensão secundária: 380 V;
    \item Taps: -4 x 0.6kV;
    \item Data de fabricação: 27/01/2021;
    \item Corrente de excitação: 2,3\%;
    \item Perdas em vazio: 1150 W;
    \item Perdas em cargas: 3850 W;
    \item Perdas totais: 5000 W;
    \item Impedância: 0,055 pu;
    \item Refrigeração: AN;
\end{itemize}

O cubículo de transformação deve ter as seguintes dimensões:

\begin{equation}
    D_{ct} = D_t + 1000mm
\end{equation}

Onde: $D_{ct}$ é a dimensão do cubículo: comprimento ($L$) ou largura ($C$), em mm; e $D_t$ é a dimensão do transformador: comprimento ou largura, em mm.

Dessa forma, consultando o catálogo do fabricante de transformadores, as dimensões foram determinadas à seguir. Ambos os cubículos de transformação levaram em conta, para o seu comprimento a menor dimensão dos transformadores correspondentes e a largura equivalente à maior dimensão do transformador correspondente. Assim para o cubículo do transformador de 300 kVA, seu comprimento será $L_{t1} = 760mm \;+\; 1000mm = 1760mm$ e largura $C_{t1} = 1270mm \;+\; 1000mm = 2270mm$. Para o cubículo do transformador de 225 kVA, seu comprimento será $L_{t1} = 725mm \;+\; 1000mm = 1725mm$ e largura $C_{t1} = 1085mm \;+\; 1000mm = 2085mm$.
