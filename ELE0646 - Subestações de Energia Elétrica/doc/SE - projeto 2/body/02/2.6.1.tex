
\paragraph{Barramento:}

O barramento será em barra de cobre de 25 mm² de seção conforme o Quadro 2 da norma DIS-NOR-036 e será responsável pela conexão entre os cubículos. Os barramentos devem ser pintados de forma que: fase A seja vermelha; fase B seja branca; e fase C seja marrom. Ainda considerando os barramentos em tensão primária de distribuição, estes devem obedecer os afastamentos de acordo com o Quadro 3 da norma DIS-NOR-036, constituindo de barramentos internos com distância de 200 mm entre fases e 150 mm entre fase e neutro.

\paragraph{Diagrama unifilar:}

O diagrama unifilar da subestação se encontra nas plantas anexadas no apêndice "desenhos estruturais".

\paragraph{Chaves seccionadora:}

Serão utilizados três conjuntos de chaves seccionadoras tripolar para manobra e seccionamento sem carga e uso interno na subestação. Um conjunto entre a bucha de passagem e o disjuntor e os demais conjuntos à montante de cada transformador. Estes devem ser, também, de operação manual, de ação simultânea e com indicador mecânico de posição "ABERTO" ou "FECHADA", dotados de alavanca de manobra e suas características elétricas são:
% http://www.sarel.com.br/catalogo/files/assets/common/downloads/Sarel.pdf
\begin{itemize}
    \item Fabricante: Sarel;
    \item Modelo: S01M-SR;
    \item Corrente nominal: 400 A;
    \item Corrente suportável de nominal de curta duração: 16 kA
    \item Duração nominal da corrente suportável de curta duração: 1 segundo;
    \item Valor de crista nominal da corrente suportável: 41.6 kA;
    \item Tensão suportável de impulso (NBI): 95 kV.
\end{itemize}

\paragraph{Isolador:}

Os isoladores utilizados na subestação seguirão as seguintes especificações: tensão nominal de 15 kV; fabricante Sarel; modelo de prensa fio SI-15EPF; NBI de 95 kV; e 6 saias. Serão utilizados 5 conjuntos de 3 isoladores, totalizando 15 isoladores.

\paragraph{Bucha de passagem:}

A ligação entre o cubículo de medição com o cubículo de disjunção será por meio de 3 buchas de passagem.

% http://www.sarel.com.br/catalogo/files/assets/common/downloads/Sarel.pdf

Cada bucha de passagem será do tipo interno-interno, classe de tensão de 15 kV e tensão suportável de impulso atmosférico (NBI) de 125 kV e corrente nominal 400 A, modelo SBPP-15, fabricante Sarel.


\paragraph{Mufla terminal:}

Serão utilizados dois conjuntos de 3 muflas. Sendo um dos conjuntos com muflas de uso externo para realizarem a isolação dos cabos do poste do ponto de entrega para a rede subterrânea do ramal de entrada. O outro conjunto será utilizado na entrada subestação no cubículo de medição para a passagem de fios isolados para fios nus.

As muflas internas e externas seguirão as especificações:
% https://www.etelmaster.com.br/catalogo_energia.pdf
\begin{itemize}
    \item Fabricante: Raychem;
    \item Modelo: HVT Terminação Termocontrátil de Média Tensão para Cabos até 72 kV;
    \item Isolação até 72 kV; 
    \item Tipo: HVT-252 de 25-70 mm²;
    \item Temperatura de regime permanente de $105^{\circ}$;
    \item Temperatura de sobrecarga $130^{\circ}$;
    \item Temperatura de curto-circuito $250^{\circ}$.
\end{itemize}