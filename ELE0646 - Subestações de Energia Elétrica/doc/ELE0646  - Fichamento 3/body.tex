% Set Title, Author, and email
\title{ELE0646 - Transformadores}
\author{Levy Gabriel da S. G. \\ Engenharia elétrica - UFRN}

\maketitle
\thispagestyle{fancy}

\textbf{Divisão dos transformadores}

\begin{itemize}
    \item Finalidade:
    \begin{itemize}
        \item TRAFO de distribuição;
        \item TRAFO de força;
        \item TP e TC.
    \end{itemize}
    \item Quantidade de enrolamentos:
    \begin{itemize}
        \item Autotransformador;
        \item TRAFO de dois ou mais enrolamentos.
    \end{itemize}
    \item Formato do núcleo:
    \begin{itemize}
        \item Núcleo envolvido;
        \item Núcleo envolvente;
    \end{itemize}
    \item Quantidade de fases:
    \begin{itemize}
        \item TRAFO monofásico (MRT - monofásico com retorno pelo terra);
        \item TRAFO bifásico;
        \item TRAFO trifásico.
    \end{itemize}
    \item Meio isolante:
    \begin{itemize}
        \item TRAFO a óleo;
        \item TRAFO a seco.
    \end{itemize}
\end{itemize}

\textbf{Principais conceitos de transformadores}

\begin{itemize}
    \item Potência nominal;
    \item Tensão nominal;
    \item Derivações (\textit{tap});
    \begin{itemize}
        \item Comutador sem tensão (CST);
        \item Comutador de derivação em carga (CDC).
    \end{itemize}
    \item Corrente de curto-circuito;
    \begin{itemize}
        \item A corrente de curto-circuito de pico afeta o TRAFO em sua estrutura mecânica e a corrente de curto-circuito permanente afeta-o termicamente.
    \end{itemize}
    \item Corrente de \textit{inrush};
    \item Nível básico de isolamento (NBI);
    \item Defasamento angular (diferença angular entre as tensões de linha ou de fase dos enrolamentos de baixa e alta tensão, ou seja, $V_L = V_l \angle{\text{defasamento}}$;
    \begin{table}[H]
\centering
\begin{tabular}{|c|c|c|c|c|}
\hline
\multicolumn{2}{|c|}{Enrolamento} & Ângulo & \multicolumn{2}{c|}{Grupo de defasamento angular} \\ \hline
Alta tensão     & Baixa tensão    &        & Forma 1                  & Forma 2                \\ \hline
Estrela         & Estrela         & 0      & YNyn$0^{\circ}$                   & YNyn0                  \\ \hline
Estrela         & Estrela         & 180    & YNyn+18$0^{\circ}$                & YNyn6                  \\ \hline
Delta           & Delta           & 0      & Dd$0^{\circ}$                    & Dd0                    \\ \hline
Delta           & Delta           & 180    & Dd+18$0^{\circ}$                  & Dd6                    \\ \hline
Delta           & Estrela         & 30     & Dyn+3$0^{\circ}$                  & Dyn1                   \\ \hline
Delta           & Estrela         & -30    & Dyn-3$0^{\circ}$                  & Dyn11                  \\ \hline
Estrela         & Delta           & 30     & YNd+3$0^{\circ}$                  & YNd1                   \\ \hline
Estrela         & Delta           & -30    & YNd-3$0^{\circ}$                  & YNd11                  \\ \hline
\end{tabular}
\end{table}
\end{itemize}

\textbf{Características de desempenho do transformador}

\begin{itemize}
    \item Perdas;
    \begin{itemize}
        \item Perdas magnéticas/perdas no ferro/perdas no núcleo (constantes):
        \begin{itemize}
            \item Perdas por efeito da histerese magnética;
            \item Perdas por correntes parasitas ou correntes de Foucault (usar material ferromagnético de alta resistividade e laminar o núcleo e realizar o isolamento das chapas);
        \end{itemize}
        \item Perdas no cobre/perdas nos enrolamentos (variáveis):
        \begin{itemize}
            \item Decorrente do efeito Joule;
        \end{itemize}
    \end{itemize}
    \item Rendimento;
    \begin{itemize}
        \item Máximo encontra-se quando as perdas no cobre forem iguais às perdas no ferro.
    \end{itemize}
    \item Regulação:
    \begin{itemize}
        \item Razão entre a diferença entre a \textbf{tensão no secundário em vazio} e a tensão no secundário sob tensão nominal e potência nominal por a tensão tensão nominal:
        \begin{equation}
            reg = \frac{|V_2^{v}|-|V_2^{pc}|}{|V_2^{pc}|}
        \end{equation}
    \end{itemize}
\end{itemize}

\textbf{Características construtivas do transformador}

\begin{itemize}
    \item Parte ativa;
    \begin{itemize}
        \item Núcleo;
        \item Enrolamentos;
        \item Comutador de derivações.
    \end{itemize}
    \item Buchas;
    \begin{itemize}
        \item Corpo isolante;
        \item Condutor passante;
        \item Terminal;
        \item Vedação.
    \end{itemize}
    \item Tanque;
    \begin{itemize}
        \item Selado;
        \item Com conservador de óleo (tanque de expansão);
    \end{itemize}
    \item Radiador;
    \begin{itemize}
        \item Tubular;
        \item Chapa de aço;
    \end{itemize}
    \item Líquido isolante e resfriamento;
    \begin{itemize}
        \item À óleo;
        \begin{itemize}
            \item Garantir isolação;
            \item Dissipar calor gerado;
            \item Óleo mineral ou de silicone
        \end{itemize}
        \item À seco;
    \end{itemize}
    \item Acessórios;
    \begin{itemize}
        \item Indicador de nível de óleo;
        \item Termômetro para indicação da temperatura do óleo;
        \item Imagem térmica;
        \item Válvula de alívio de pressão;
        \item Relé de pressão súbita;
        \item Relé de gás (Buchholz);
        \item Secador de ar.
    \end{itemize}
\end{itemize}

\textbf{Tipos de resfriamento}

\begin{itemize}
    \item ONAN: óleo natural, ar natural;
    \item ONAF: óleo natural, ar forçado;
    \item OFAN: óleo forçado, ar natural;
    \item OFAF: óleo forçado, ar forçado;
    \item OFWF: óleo forçado, água forçada;
    \item AN: ar natural (destinado a transformadores a seco);
    \item AF: ar forçado (destinado a transformadores a seco);
\end{itemize}