\section{Determinação dos ajustes das unidades 59 e 47}

A unidade 59 (sobretensão) recebe informações de dois ou três TPs (13800/115). Então, deverá ser graduada com base na tensão secundária (115 V) dos TPs.

No caso da unidade 47 (sequência de fases), basta ativá-la nos parâmetros do relé (ver software de graduação do relé).

\begin{comment}
Neste estudo, o relé de subtensão (função 27) será conectado à rede de 13,8 kV por meio de dois TPs (13800/220 →RTP = 62,7), 100 VA, ligados em delta aberto ou “V”. Então, o relé irá monitorar a tensão de linha do sistema. Por sugestão da NORMA DISTRIBU-ENGE-0023/COSERN, essa unidade poderá ser graduada para atuar quando a tensão do sistema “cair” para um valor igual ou menor do que 0,85 p.u, ou seja, igual ou menor do que 11,7 kV. Por segurança, será graduado para atuar quando a tensão for igual ou menor do que 11,5 kV (valor primário).

A graduação do tempo de atuação do relé de subtensão (função 27) levará em conta o tempo acumulado da sequência de operação do religador de retaguarda (proteção geral do alimentador da Concessionária), de modo a assegurar que não haja atuação indesejada dessa proteção durante o ciclo de operação normal do religador, pois o relé iria desligar a respectiva subestação desnecessariamente. Então, de acordo com essas considerações, o tempo de atuação do relé será graduado em 12 s (tempo definido).    
\end{comment}
