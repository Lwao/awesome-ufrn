% assinatura

\begin{center}
    \mbox{}
    \vfill
    
    
    \begin{figure}[h!]
        \centering
        \includegraphics[width=6cm]{images/ass.png}
        \label{fig:ass}
    \end{figure}
    \noindent\rule{8cm}{0.4pt} \\
    \vspace{0.25cm}
    Eng. Eletricista Levy Gabriel da Silva Galvão \\
    Crea-RN nº 20170056839
    
    \vspace{2.5cm}
    
    Natal, RN. Janeiro de 2022.

\end{center}


\newpage

\section{Apêndice}

\subsection{Circuitos de sequências para o sistema proposto}

As figuras \ref{ckt:1}, \ref{ckt:2} e \ref{ckt:3} apresentam o três circuitos de sequências para o sistema proposto, i.e. circuito de sequência positiva, negativa e zero, respectivamente. Estes definem a simbologia a ser utilizada durante os cálculos de curto circuito no ponto de entrega (PE) e no lado de baixa tensão (BT) do transformador.

\begin{figure}[H]
\centering

\tikz \node [scale=0.95, inner sep=0] {
\begin{tikzpicture} [american]
    \draw (0,0) to[vsourcesin, l=$V_{1th}$] (0,3)
    (0,3) to[generic, l=$Z_{1th}$] (3,3)
    (3,3) to[generic, l=$Z_{1trafo}$] (6,3)
    (5.5,3) to[short, l=$I_1$, i=$ $] (7,3) 
    (0,0) -- (7,0)
    (3,2.5) -- (3,3.5) 
    (7,2.5) -- (7,3.5);
    \node[black] at (3,3.8) {PE};
    \node[black] at (7,3.8) {BT};
\end{tikzpicture}
};

\caption{Circuito de sequência positiva do sistema proposto.}
\label{ckt:1} 
\end{figure}

\begin{figure}[H]
\centering

\tikz \node [scale=0.8, inner sep=0] {
\begin{tikzpicture} [american]
    \draw[>=triangle 90, ->] (6.5,0) -- (5.5,0);
    \node[black] at (7,0) {$V_{noise}$}; 
    \draw  (4,0) to[R, l=$R_L$, *-] (4,-4) node[ground]{}
    (2,0) to[R, l=$R_T$, -] (2,-4) node[ground]{}
    (2,0) -- (5,0)
    ;
\end{tikzpicture}
};

\caption{Circuit setup to measure noisy resistor voltage in a combination of resistors. Source: own.}
\label{ckt:2} 
\end{figure}

\begin{figure}[H]
\centering

\tikz \node [scale=0.8, inner sep=0] {
\begin{tikzpicture} [american]
    \draw[>=triangle 90, ->] (6.5,0) -- (5.5,0);
    \node[black] at (7,0) {$V_{noise}$}; 
    \draw  (4,0) to[R, l=$C$, *-] (4,-4) node[ground]{}
    (2,0) to[R, l=$R$, -] (2,-4) node[ground]{}
    (2,0) -- (5,0)
    ;
\end{tikzpicture}
};

\caption{Circuit setup to measure noisy resistor voltage in a RC circuit with a ideal capacitor. Source: own.}
\label{ckt:3} 
\end{figure}

\subsection{Correntes de curto-circuito no ponto de entrega}

Considera-se: 

\begin{flushleft}
$Z_{base_{AT}} = \frac{V_{base_{AT}}^2}{S_{base}} = 1.9044 \; \Omega$ \\ \vspace{5pt}
$I_{base_{AT}} = \frac{S_{base}}{\sqrt{3}V_{base_{AT}}} = 4183.7 \; A$ 
\end{flushleft}

Assim as correntes são:

\begin{flushleft}
$I_{CC3F_{AT}} = \left|\frac{1}{Z_{1th}}\right| \cdot I_{base_{AT}} = 2271 \; A$ \\
$I_{CC2F_{AT}} = \left|\frac{j\sqrt{3}}{Z_{1th}+Z_{2th}}\right| \cdot I_{base_{AT}} = 1967 \; A$ \\
$I_{CCFT_{AT}} = \left|\frac{3}{Z_{1th}+Z_{2th}+Z_{0th}}\right| \cdot I_{base_{AT}} = 1785 \; A$ \\
$I_{CCFTm_{AT}} = \left|\frac{3}{Z_{1th}+Z_{2th}+Z_{0th}+3\frac{R_t}{Z_{base_{AT}}}}\right| \cdot I_{base_{AT}} = 188 \; A$, $R_t = 40 \; \Omega$ 
\end{flushleft}

\subsection{Correntes de curto-circuito no secundário do transformador}

As grandezas de base devem ser ajustadas para o secundário, assim considera-se:

\begin{flushleft}
$Z_{base_{BT}} = \frac{V_{base_{BT}}^2}{S_{base}} = 0.001444 \; \Omega$ \\ \vspace{5pt}
$I_{base_{BT}} = \frac{S_{base}}{\sqrt{3}V_{base_{BT}}} = 151934.28137 \; A$ 
\end{flushleft}

Neste caso, além das impedâncias do ponto de entrega, também deve ser considerada a impedância $z=j5.7\%$ do transformador que é dada em sua base ($S_{base_{TRAFO}} = 500 \; kVA$, $V_{base_{AT}} = 13.8 \; kV$). Esta deve ser convertida para a base do sistema, assim tem-se:

\begin{flushleft}
$Z = z \cdot \frac{V_{base_{AT}}^2}{S_{base_{TRAFO}}} \cdot \frac{1}{Z_{base_{AT}}} = j11.4 \; p.u$ 
\end{flushleft}

Considerando que todas as reatâncias dos circuitos de sequência são as mesmas que a encontrada anteriormente, tem-se que: $Z = Z_{1trafo} = Z_{2trafo} = Z_{0trafo}$.

Finalmente a correntes serão:

\begin{flushleft}
$I_{CC3F_{BT}} = \left|\frac{1}{Z_{1th}+Z}\right| \cdot I_{base_{AT}} = 11636 \; A$ \\
$I_{CC2F_{BT}} = \left|\frac{j\sqrt{3}}{Z_{1th}+Z_{2th}+2Z}\right| \cdot I_{base_{BT}} = 10077 \; A$ \\
$I_{CCFT_{BT}} = \left|\frac{3}{Z_{1th}+Z_{2th}+3Z}\right| \cdot I_{base_{BT}} = 12155 \; A$ \\
$I_{CCFTm_{BT}} = \left|\frac{3}{Z_{1th}+Z_{2th}+3Z+3\frac{R_t}{Z_{base_{BT}}}}\right| \cdot I_{base_{BT}} = 43.87 \; A$, $R_t = 5 \; \Omega$ 
\end{flushleft}

\subsection{Correntes de curto-circuito referidas ao primário do transformador}

A corrente de curto-circuito trifásica referida ao lado de alta tensão pode ser obtida a partir da relação de transformação das tensões:

\begin{flushleft}
$I_{CC3F_{AT}} = I_{CC3F_{BT}} \cdot \frac{380}{13800} = 320.4 \; A$ 
\end{flushleft}

As demais correntes de curto-circuito, i.e. bifásica, fase-terra e fase-terra mínimo referida ao lado de alta tensão podem ser obtidas a partir da relação de espiras:

\begin{flushleft}
$I_{CC2F_{AT}} = 2I_{CC2F_{BT}} \cdot \frac{220}{13800} = 321.28\; A$  \\ \vspace{5pt}
$I_{CCFT_{AT}} = I_{CCFT_{BT}} \cdot \frac{220}{13800} = 193.78 \; A$  \\ \vspace{5pt}
$I_{CCFTm_{AT}} = I_{CCFTm_{BT}} \cdot \frac{220}{13800} = 0.7 \; A$  \\
\end{flushleft}

\subsection{Dimensionamento do TC}

Considerando a corrente de carga baseada na demanda máxima determinada na seção que lista os dados da subestação do consumidor de: $I_{carga,max} = 17.57 \; A$; permite-se limitar a corrente primária do TC a ser maior que a de carga modificada do fator de crescimento ($k = 1.1$), de forma que:

\begin{center}
$I_{prim,TC} \ge  k \cdot I_{carga,max}$ \\
$I_{prim,TC} \ge 1.1 \cdot 17.57 \; A$ \\
$I_{prim,TC} \ge 19.33 \; A$ 
\end{center}

A corrente nominal primária do TC também deve ser maior do que a razão entre o curto-circuito máximo no ponto de entrega ($I_{CC3F_{AT}} = 2271 \; A$) e o fator de sobrecorrente do TC ($FS=20$), assim:

\begin{center}
$I_{prim,TC} \ge  \frac{I_{CC3F_{AT}} }{FS}$ \\ \vspace{5pt}
$I_{prim,TC} \ge \frac{2271}{20}$ \\ \vspace{5pt}
$I_{prim,TC} \ge 114 \; A$ 
\end{center}

Dessa forma, permite-se determinar um TC de proteção 150/5 com relação de transformação de corrente $RTC=30$. 

O \textit{burden} do TC será considerado aproximadamente de $300 \; m\Omega$, baseando-se no exemplo fornecido pelo texto da norma DIS-NOR-036. Isso garante que, para a maior corrente de curto-circuito no lado de alta tensão, permite definir a tensão máxima do secundário do TC como:

\begin{center}
$V_{max,TC} = \frac{I_{CC3F_{AT}}}{RTC} \cdot 300 \cdot 10^{-3}$ \\ \vspace{5pt}
$V_{max,TC} = \frac{2271}{30} \cdot 300 \cdot 10^{-3}$ \\ \vspace{5pt}
$V_{max,TC} = 22.71 \; V$ 
\end{center}

Por fim, permite dimensionar um TC com precisão 10B50, que possui $V_{max,TC} = 50 \; V$.


\subsection{Ajustes das unidades de sobrecorrente de fase e neutro do relé}

\subsubsection{Unidade 51}

\begin{comment}
Para graduação desta unidade, a corrente mínima de atuação no secundário do transformador ($TAP$) deve ser maior que a maior corrente de carga no secundário e ajustada pelo fator de crescimento; e menor que a corrente de curto-circuito bifásico dentro da sua zona de proteção (lado de alta do transformador), portanto:
\end{comment}

Para a graduação desta unidade a corrente mínima de atuação no secundário do transformador ($TAP$) deve atender os seguintes critérios:

\begin{center}
$\frac{k \cdot I_{carga,max}}{RTC} \leq TAP \leq \frac{I_{CC2F_{AT}} }{RTC}$ \\ \vspace{5pt}
$\frac{1.1 \cdot 17.57}{30} \leq TAP \leq \frac{321}{30}$ \\ \vspace{5pt}
$0.64 \leq TAP \leq 10.7$ \\ 
\end{center}

O valor escolhido é $TAP = 0.68 \; A$. 

De acordo com os ajustes estabelecidos pela concessionária para a unidade 51, procura-se estabelecer um tempo de atuação da unidade 51 da SE CONSUMIDOR 0.2 segundos mais rápido que o relé da concessionária, para a maior corrente de curto-circuito, portanto:

\begin{center}
$t_{51} \geq \frac{k_1 \cdot TMS}{\left(\frac{I_{CC3F_{AT}}}{RTC_{con} \cdot TAP_{con}}\right)^{k_2}-1} - 0.2$ \\ \vspace{5pt}
$t_{51} \geq \frac{13.5 \cdot 0.15}{\frac{2271}{120 \cdot 2.8}-1} - 0.2$ \\ \vspace{5pt}
$t_{51} \geq 0.15163 \; s$ \\ \vspace{5pt}
\end{center}

Agora aplicando aos parâmetros da SE CONSUMIDOR para a mesma corrente para obter a curva adequada:

\begin{center}
$t_{51} \geq \frac{k_1 \cdot TMS}{\left(\frac{I_{CC3F_{AT}}}{RTC \cdot TAP}\right)^{k_2}-1}$ \\ \vspace{5pt}
$TMS \leq \frac{t_{51}}{k_1} \left(\left(\frac{I_{CC3F_{AT}}}{RTC \cdot TAP}\right)^{k_2}-1\right)$ \\ \vspace{5pt}
$TMS \leq \frac{0.15163}{13.5} \left(\left(\frac{2271}{30 \cdot 0.68}\right)^{k_2}-1\right)$ \\ \vspace{5pt}
$TMS \leq 1.23$ \\ \vspace{5pt}
\end{center}

Portanto, define-se a curva 0.9-MI-IEC para a unidade 51 do relé da SE CONSUMIDOR.


\subsubsection{Unidade 50}

\begin{comment}
Para a graduação desta unidade a corrente mínima de atuação no secundário do transformador ($TAP$) deve superar a corrente máxima de carga escalonada por um fator de multiplicação (esse fator pode ser de 3 até 8, o valor máximo é utilizado para casos em que existem muitos transformadores e motores de indução), que será escolhido como 8, considerando a presença de diversos motores de indução devido aparelhos de condicionamento de ar. O $TAP$ também deve ser maior que a corrente de curto-circuito trifásico próximo ao relé, portanto:
\end{comment}

Para a graduação desta unidade a corrente mínima de atuação no secundário do transformador ($TAP$) deve atender os seguintes critérios:

\begin{center}
$TAP \geq \frac{8 \cdot I_{carga,max}}{RTC}$ \\ \vspace{5pt}
$TAP \geq \frac{8 \cdot 17.57}{30} $ \\ \vspace{5pt}
$TAP \geq  4.7$ \\ \vspace{5pt}
\end{center}

e

\begin{center}
$TAP \geq \frac{I_{CC3F_{AT}}}{RTC}$ \\ \vspace{5pt}
$TAP \geq \frac{320}{30} $ \\ \vspace{5pt}
$TAP \geq  10.7$ \\ \vspace{5pt}
\end{center}

O fator de multiplicação foi escolhido como 8, considerando a presença de diversos motores de indução devido aparelhos de condicionamento de ar.

O valor escolhido é $TAP = 13 \; A$. 

\subsubsection{Unidade 51NS}

A proteção 51NS deve ser do tipo tempo definido. Seu ajuste será definido a uma corrente de 40\% da unidade 51NS da concessionária, esta sendo 12 A referida ao primário, o que resulta em um ajuste de 4.8 A para a unidade do consumidor.

O tempo de atuação deve ser no mínimo 0.4 segundos mais rápido que a da concessionária. Como o da concessionária foi definido como 5 segundos, será escolhido um tempo de atuação de 3 segundos para a unidade 51NS do consumidor.

\subsubsection{Unidade 51N}

Para a graduação desta unidade a corrente mínima de atuação no secundário do transformador ($TAP$) deve atender os seguintes critérios:

\begin{center}
$\frac{0.1 \cdot I_{carga,max}}{RTC} \leq TAP \leq \frac{I_{CCFTm_{AT}} }{RTC}$ \\ \vspace{5pt}
$\frac{0.1 \cdot 17.57}{30} \leq TAP \leq \frac{188}{30}$ \\ \vspace{5pt}
$0.058567 \leq TAP \leq 6.266667$ \\ \vspace{5pt}
\end{center}

O curto-circuito é considerado no final do trecho, portanto no ponto de entrega.

O TAP da unidade 51N do consumidor deve ser menor que o TAP da unidade 51NS da concessionária ($TAP < 12/RTC_{con} \longrightarrow TAP < 0.4 \; A$, com $RTC_{con}=30$). Portanto o valor escolhido é $TAP = 0.2 \; A$. 

Como o múltiplo avaliado na corrente de curto fase-terra franco no ponto de entrega é deveras alto ($m = I_{CCFT_{AT}}/(RTC \cdot TAP) = 1785/(30 \cdot 0.2) = 297.50$), este deve ser limitado a $m=20$. Isso resulta na corrente limite superior da faixa de coordenação dada por: $I_{lim} = m \cdot RTC \cdot TAP = 20 \cdot 30 \cdot 0.2 = 180 \; A$. Esta corrente será utilizada para coordenar as unidades 51N.

De acordo com os ajustes estabelecidos pela concessionária para a unidade 51N, procura-se estabelecer um tempo de atuação da unidade 51N da SE CONSUMIDOR 0.2 segundos mais rápido que o relé da concessionária, para a corrente limitada, anteriormente encontrada, tem-se:

\begin{center}
$t_{51N} \geq \frac{k_1 \cdot TMS}{\left(\frac{I_{lim}}{RTC_{con} \cdot TAP_{con}}\right)^{k_2}-1} - 0.2$ \\ \vspace{5pt}
$t_{51N} \geq \frac{13.5 \cdot 0.2}{\frac{180}{120 \cdot 0.5}-1} - 0.2$ \\ \vspace{5pt}
$t_{51N} \geq 0.475 \; s$ \\ \vspace{5pt}
\end{center}

Agora aplicando aos parâmetros da SE CONSUMIDOR para a mesma corrente para obter a curva adequada:

\begin{center}
$t_{51N} \geq \frac{k_1 \cdot TMS}{\left(\frac{I_{lim}}{RTC \cdot TAP}\right)^{k_2}-1}$ \\ \vspace{5pt}
$TMS \leq \frac{t_{51}}{k_1} \left(\left(\frac{I_{lim}}{RTC \cdot TAP}\right)^{k_2}-1\right)$ \\ \vspace{5pt}
$TMS \leq \frac{0.475}{13.5} \left(\left(\frac{180}{30 \cdot 0.2}\right)^{k_2}-1\right)$ \\ \vspace{5pt}
$TMS \leq $ 0.67\\ \vspace{5pt}
\end{center}

Portanto, define-se a curva 0.2-MI-IEC para a unidade 51N do relé da SE CONSUMIDOR.

\subsubsection{Unidade 50N}

Para a graduação desta unidade a corrente mínima de atuação no secundário do transformador ($TAP$) deve atender os seguintes critérios:

\begin{center}
$TAP \geq \frac{8 \cdot 0.3 \cdot I_{carga,max}}{RTC}$ \\ \vspace{5pt}
$TAP \geq \frac{8 \cdot 0.3 \cdot 17.57}{30} $ \\ \vspace{5pt}
$TAP \geq  1.4$ \\ \vspace{5pt}
\end{center}

e

\begin{center}
$TAP \geq \frac{I_{CCFT_{AT}}}{RTC}$ \\ \vspace{5pt}
$TAP \geq \frac{\cdot 194}{30} $ \\ \vspace{5pt}
$TAP \geq 6.5$ \\ \vspace{5pt}
\end{center}    

O fator de multiplicação de 0.3 foi escolhido para compensar o desequilíbrio.

O valor escolhido é $TAP = 8 \; A$.

\subsubsection{Unidade 59}

A unidade 59 será graduada para atuar quando a tensão do sistema for 120\% da tensão nominal. Considerando a tensão nominal como a do secundário do TP (115 V), a graduação da unidade 59 será 138 V. Isso corresponde ao \textit{pickup} de: $RTP \cdot 138 = 13800/115 \cdot  138 = 16.5 \; kV$.

\newpage



\newpage