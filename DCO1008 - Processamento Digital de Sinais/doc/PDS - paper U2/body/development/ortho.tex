\subsection{Orthogonality and Discrete Fourier Transform}

Two important concepts in terms of OFDM are the orthogonality of waveforms and the relation of the discrete fourier transform with such waveforms. 

\par Extending the inner product between vectors of linear algebra it is possible to define the concept of orthogonality between two real signals $\phi_1(t)$ and $\phi_2(t)$ as follows:

\begin{equation} \label{eq:ortho_eq}
    \int_{-\infty}^{\infty} \phi_1(t)\phi_2(t)dt = 0
\end{equation}

\par From the definition in equation \ref{eq:ortho_eq} it's possible demonstrate the orthogonality between the set of complex exponentials pairwise. 

\begin{equation} \label{eq:ortho_exp_eq}
    \frac{1}{N}\sum_{n = 0}^{N-1} e^{j(2\pi/N)(k-r)n} =
    \left\{
	\begin{array}{ll}
		1  & \mbox{if } k-r = mN,  \quad m \in \mathbf{Z} \\
		0  & \mbox{otherwise. } 
	\end{array}
\right.
\end{equation}


\par A simple way to visualize this result is looking in a time period, each subcarrier experiments a integer number of cycles and remembering that in such case the area over the curve is zero. 

\par This inherent orthogonality of OFDM can also be observed in the frequency domain, since each OFDM symbol is the convolution of N impulses with a rectangular pulse of amplitude 1 and duration of $T_N$ and, therefore, the spectrum of an OFDM symbol is a set of N $sinc_i(\pi f_i T_N)$ functions whose zeros intersect the centers of each other  $sinc_j(\pi f_j T_N)$ when $i \neq j$.

\par Thus, considering that there aren't frequency deviations, there will be no ISI between symbols transmitted on different carriers, therefore will be no ICI.

