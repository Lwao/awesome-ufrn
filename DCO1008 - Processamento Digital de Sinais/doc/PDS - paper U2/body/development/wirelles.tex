\subsection{Wireless Channel Characteristics}

\par In conventional communication systems where the channel frequency response is approximately flat for a certain frequency range, typically a narrower range, single carrier modulations are commonly used. In these modulations schemes a data sequence (modulated at the symbol rate of the source) is sending over the channel in a single carrier. However in situations where the symbol rate are very high and the transmission medium is the air, some problems may occur.

\par One of the most important problems it's the presence of multipath components. Since the propagation path between transmitter and receiver contains objects that reflect, refract or spread the transmitted signal, in receiver the effect of these components translates as the reception of copies of transmitted signal with different delay, attenuation and phase shifts. The sum of these multiples copies causes variations on the instantaneous power of the received signal, therefore producing fading.

\par As a consequence of the internet popularization, there is a rising demand of higher data rates and to achieve this, one of the solutions is to decrease the symbol time period. However, this solution increase the signal bandwidth and if channel coherent bandwidth is smaller than the signal bandwidth then the channel creates a frequency selective fading as a direct consequence, the spectral response of the signal will show dips due to the multipath. In addition, this type of fading is also dispersive, since the signal energy associated with each symbol is spread out in time. This causes transmitted symbols that are adjacent in time to interfere with each other and this phenomenon is called intersymbol interference (ISI).
