\textbf{Faça um esboço do Espectro de Magnitude do sinal amostrado por um trem de impulso.}

Considerando o espectro de um sinal de tempo contínuo $X(\omega)$ como:

\begin{figure}[H]
\centering
\tikz \node [scale=0.8, inner sep=0] {
\begin{tikzpicture} 
    \draw[black, very thick] (-2,0) -- (0,0);
    \draw[black, very thick] (0,0) -- (0,3);
    \draw[black, very thick] (0,3) -- (2,3);
    \node[blue] at (-2.4,0) {$R_{in}$};  
    \node[red] at (2.4,3) {$R_{L}$};  
    \node[black] at (1,1.5) {$1+Q_p^2$};  
    \draw[>=latex, <->] (0.2,0) -- (0.2,3);
    \node[black] at (0,-0.5) {(a)};
\end{tikzpicture}
};
\hspace{1cm}
\tikz \node [scale=0.8, inner sep=0] {
\begin{tikzpicture} [american]
    \draw[>=triangle 90, ->] (-0.5,0) -- (0.5,0);
    \draw[>=triangle 90, ->] (6,0) -- (7,0);
    \node[black] at (-0.9,0) {$R_{in}$}; 
    \node[black] at (5.6,0) {$R_L$}; 
    \node[black] at (3,-4) {(b)};
    \draw (1,0) to[C, l=$C$, *-*] (5,0)
    (4,0) to[L, l=$L$, *-] (4,-3) node[ground]{}
    ;
\end{tikzpicture}
};

\caption{(a) Desired effect of impedance gain (b) L matching network for a parallel to serial transformation of load impedance. Source: own.}
\label{graph:1} 
\end{figure}

De acordo com a equação \ref{eq1:4}, o espectro de magnitude deste sinal amostrado por um trem de impulsos será:

\begin{figure}[H]
\begin{center}
\begin{tikzpicture} 
\tikzset{every pin edge/.style={scale=0.00001}    }
\begin{axis}[very thick,
                     samples = 100,
                     ytick={-0.25,2},
                     xlabel = {$t$},
                     ylabel = {$y(t)$},
                     xmin = -1,
                     xmax = 7,
                     ymin = -0.25,
                     ymax = 1.5,
                     axis x line = middle,
                     axis y line = middle,
                     ticks = none]
                     
            % waveform
            \addplot[mark=none] coordinates {(0+2,0) (1+2,1)};
            \addplot[mark=none] coordinates {(1+2,1) (3+2,0)};
            
            \addplot[mark=none] coordinates {(0+1,0) (1+1,1-0.5)};
            \addplot[mark=none] coordinates {(1+1,1-0.5) (3+1,0)};
            
            \addplot[mark=none] coordinates {(0+3,0) (1+3,1-0.5)};
            \addplot[mark=none] coordinates {(1+3,1-0.5) (3+3,0)};
            
            % labels
            \addplot[mark=none] coordinates {(1,0.2)} node[pin=270:{\scriptsize$t_d-T$}]{};
            \addplot[mark=none] coordinates {(2,0.2)} node[pin=270:{\scriptsize$t_d$}]{};
            \addplot[mark=none] coordinates {(3,0.2)} node[pin=270:{\scriptsize$t_d+T$}]{};
            \addplot[mark=none] coordinates {(0.8,1)} node[pin=180:{$A$}]{};
            \addplot[mark=none] coordinates {(0.9,0.5)} node[pin=180:{$A/2$}]{};
            
            % dashed line
            \addplot[dashed] coordinates {(0,1) (3,1)};
            \addplot[dashed] coordinates {(0,0.5) (4,0.5)};
            
            
        \end{axis}
\end{tikzpicture}
\end{center}
\caption{Forma de onda do sinal $y(t)$, como uma composição de sinais $g(t)$, com : $t_d>T$.}
\label{graph:2} 
\end{figure}

Observa-se claramente pela figura \ref{graph:2} que a frequência de amostragem foi corretamente escolhida de acordo com o teorema de Nyquist-Shannon, ou seja, $\omega_s>2\pi B$. 
