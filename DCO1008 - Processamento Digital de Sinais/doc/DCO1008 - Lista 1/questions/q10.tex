\textbf{Como a distorção produzida pelo ROZ pode ser mitigata?}
 
 A distorção produzida pelo ROZ pode ser mitigada por uma filtro equalizador. Enquanto que a resposta em frequência do ROZ é (onde $T_p$ é a duração de pulso):
 
 \begin{equation} \label{eq10:1}
     h_0(t) ) \rightleftharpoons H_0(\omega) = T_p \text{sinc}\left(\frac{\omega T_p}{2}\right) e^{-j\frac{\omega T_p}{2}}
 \end{equation}
 
 A partir da resposta em frequência, observa-se que o ROZ distorce o sinal por meio de: uma defasagem linear no tempo, proporcionada pela exponencial que depende de $T_p/2$; distorção do sinal de saída na sua banda de passagem; e lóbulos menores vestigiais em frequências múltiplas da frequência de amostragem, permitindo que as cópias não desejadas inerentes ao processo de digitalização possuam magnitude diferente de zero.
 
Esses efeitos podem ser mitigados por meio de um filtro equalizador. Este filtro tenta desfazer as distorções implementadas pelo ROZ, assim sua resposta em frequência pode seguir:

\begin{equation} \label{eq10:2}
\centering
E(\omega) =
\left \{
\begin{array}{cc}
T_s/H_0(\omega), & |\omega| \leq 2\pi B \\
\text{Flexível}, & 2\pi B < |\omega| < (\omega_s - 2\pi B) \\
0, & |\omega| \geq (\omega_s - 2\pi B)
\end{array}
\right.
\end{equation}

A equação \ref{eq10:2} mostra que a resposta em frequência do equalizador procura desfazer as distorções do ROZ durante toda a banda de passagem do sinal $[-2\pi B, 2\pi B]$. O equalizador possui uma resposta flexível entre a banda de passagem e a próxima cópia, permitindo ajustar a seletividade e definir especificações de \textit{ripple} da banda de corte, seletividade do filtro, etc. Por fim é desejável que além das cópias do espectro do sinal original a resposta seja nula. 

Também é importante manter a resposta da banda de passagem do equalizador realizável. Isto é, outro atraso temporal pode ser adicionado. Trabalhar com pulsos curtos é sempre desejável, pois quanto menor estes mais aproximada a resposta do equalizador será constante, como pode ser visto abaixo para uma pequena largura de pulso $T_p$:

\begin{equation}
    E(\omega) = T_s \frac{\omega/2}{\text{sin}(\omega T_p/2)} e^{-j\omega t_0} = \frac{T_s}{T_p} e^{-j\omega t_0}
\end{equation}
