\textbf{De que maneira o filtro antialiasing influencia no sistema de amostragem?}

Uma vez que sinais reais são limitados no tempo, consequentemente ilimitados em banda, inevitavelmente quando amostrados ocorrerá algum tipo de \textit{aliasing} no sinal resultante. O efeito do \textit{aliasiang} pode ser mitigado realizando uma pré-filtragem passas-baixas no sinal analógico antes do processo de amostragem. Essa filtragem é realizada pelo filtro \textit{antialiasiang} e permite que o sinal a ser amostrado possua uma banda aproximadamente limitada dentro da faixa do sinal recebido, evitando os efeitos do \textit{aliasing}.



