\textbf{O que significa reamostrar um sinal?}

A re amostragem de um sinal significa alterar a sua taxa de amostragem para valores maiores ou menores que a taxa atual. Esta pode ser feita por meio de uma subamostragem, que consiste no descarte de amostras a uma determinada taxa, assim reduzindo o tamanho efetivo da sequência a ser processada e resulta em uma taxa de amostragem maior, devido a compressão do sinal no tempo. A superamostragem atua de maneira complementar, inserindo amostras nulas entre amostras da sequência resultando em uma sequência com maior e com uma taxa de amostragem menor devido à expansão no tempo e compressão do espectro de frequência. As amostras nulas inseridas na superamostragem podem passar por uma interpolação para que as descontinuidades sejam eliminadas.
