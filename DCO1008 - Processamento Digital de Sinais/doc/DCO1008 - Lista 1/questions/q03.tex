\textbf{O que é \textit{aliasing}?}

O \textit{aliasing} ocorre quando o teorema da amostragem não é obedecido, ou seja, a frequência de amostragem é escolhida tal que $\omega_s<2\pi B$. Quando este ocorre, observa-se pelo espectro de magnitude do sinal amostrado que as versões do espectro de magnitude do sinal original sobrepõe umas sobre as outras. O \textit{alisiang} é tão prejudicial quanto menor for a frequência de amostragem abaixo do limiar de Nyquist, pois haverá uma maior faixa do espectro interferindo sobre suas cópias.

