\section{Conclusões}

O controle digital por computador se mostrou uma boa alternativa ao controle analógico tradicional, pois ele permite a facilidade de reprojeto e um ambiente com condições externas controladas, evitando ter que se preocupar com fatores limitantes do ambiente ou de tolerância de componentes. Porém, como a representação digital é apenas uma aproximação da representação analógica, a velocidade no processamento dos dados será um barreira limitante, pois ela dita a precisão do modelo. Entretanto, ao obedecer o teorema da amostragem e conhecimento prévio do hardware no qual o controlador será implementado, as barreiras limitantes diminuem, permitindo a expansão das técnicas de controle tradicionais para novos horizontes de aplicações.
\pagebreak