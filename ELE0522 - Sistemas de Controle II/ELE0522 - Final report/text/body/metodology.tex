\section{Metodologia}

Para aplicar os controladores na prática e avaliar os resultados foram seguidos os seguintes passos:

\begin{itemize}
    \item Consulta da literatura vigente em projeto de circuitos eletrônicos e sistemas de controle para projetar os circuitos;
    \item Simular os circuitos que representam a planta em malha aberta e fechada e a planta com controlador em malha fechada em um \textit{software} de simulação para verificar a veracidade do projeto;
    \item Implementar o circuito na prática e fazer testes em laboratório e coletar os dados.
\end{itemize}

Os circuitos analógicos foram projetados baseados em amplificadores operacionais (AMPOPS) para modelar a planta, controlador PD, controlador por realimentação de estados e o subtrator da realimentação negativa. As topologias de circuitos utilizadas são aquelas propostas por OGATA(2009), KUO(1995) e SEDRA(2016).

Após o projeto dos circuitos, estes foram inicialmente simulados e otimizados no LTSpice para depois serem montado na prática.

Durante a realização das práticas, todos os protocolos de segurança dos discentes e proteção dos equipamentos foram devidamente seguidos. No que tange a montagem dos circuitos, as fontes de ruído foram evitadas, tais como ponteiras de medição enferrujadas, \textit{protoboards} danificados, etc. Os componentes dos circuitos foram devidamente testados antes da montagem, garantindo que os componentes passivos, tais como resistores e capacitores utilizados, obedeciam a faixa de tolerância oferecida pelo fabricante.

Dessa forma os circuitos foram alimentados e as saídas de tensão foram observadas no osciloscópio, gerando material para ser discutido nos resultados.

\pagebreak
