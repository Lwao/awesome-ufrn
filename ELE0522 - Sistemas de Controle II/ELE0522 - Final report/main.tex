\documentclass[a4paper, 12pt]{article}


\usepackage{graphicx}			% usar gráficos
\usepackage{multicol,lipsum}	%
\usepackage[utf8x]{inputenc}		% Codificacao do documento (conversão automática dos acentos)
\usepackage{indentfirst}		% Indenta o primeiro parágrafo de cada seção.
\usepackage{color}				% Controle das cores
\usepackage{graphicx}			% Inclusão de gráficos
\usepackage{graphics}           % Inclusão de gráficos
\usepackage{microtype} 			% para melhorias de justificação
\usepackage{natbib} 			% citar
\usepackage{float}				% números em ponto flutuante
\usepackage[brazil]{babel}		% língua brasileira
\usepackage{setspace}			% espaçamento entre linhas
\usepackage{subfigure}          % Permite subfiguras
\usepackage{pgfplots}           % Realizar plots
\usepackage{pdfpages}           % Inserir páginas de PDF como figura
\usepackage{comment}            % Realizar múltiplos comentários
\usepackage{listings}           % Código fonte
\usepackage{lipsum}             % Gerador de Lorem Ipsum
\usepackage{multirow}           % Multilinhas em uma tabela
\usepackage{mathtools}

\graphicspath{{images/}}        % Caminho das imagens
\pagenumbering{arabic}          % Numeração das páginas
\everymath{\displaystyle}       % \frac{}{} tamanho ideal

\usepackage{amsfonts}  % Matemática 
\usepackage{amsmath}
\usepackage{amssymb}
\usepackage{caption}
\usepackage[amssymb]{SIunits}
\usepackage[pdftex]{graphicx}




% Outros pacotes
\usepackage{fullpage}
\usepackage{euscript}
\usepackage[colorlinks=true, linkcolor=blue, urlcolor=blue, pdfborder={0 0 0}]{hyperref}
\usepackage[top=100pt,bottom=100pt,left=68pt,right=66pt]{geometry}
\usepackage{eurosym} 
\usepackage[version=3]{mhchem} 
\usepackage{enumerate}
\usepackage{epsfig}
\usepackage{setspace}
\usepackage{pdflscape}
\usepackage[]{color}
\usepackage{alltt}
\usepackage{etoolbox}
\patchcmd{\abstract}{\null\vfil}{}{}{}
\raggedbottom
\usepackage{titlesec}
\titleformat{\chapter}
{\normalfont\LARGE\bfseries}{\thechapter.}{1em}{}
\titlespacing{\chapter}{0pt}{50pt}{2\baselineskip}
\floatstyle{plaintop}
\restylefloat{table}
\usepackage[tableposition=top]{caption}
\usepackage{hyperref}
\hypersetup{hidelinks,linkcolor = black}
\usepackage{alltt}
\usepackage{etoolbox}
\usepackage{titlesec}
\raggedbottom

%----------------------------------------------


% Fancy 
\usepackage{fancyhdr}
\fancyhf{} 
\fancyfoot[C]{\rfoot{\thepage}}     % páginas numeradas right-foot
\renewcommand{\headrulewidth}{0pt}  % retira linha horizontal do fancihdr
\pagestyle{fancy}
%-------------------------------------
% Tikz
\usepackage{tikz}               
\usepackage{tikzscale}          
\usepackage[siunitx,american,cuteinductors,smartlabels]{circuitikz}
\usepackage{bodegraph}
\usetikzlibrary{intersections}
\usetikzlibrary{calc}
\usetikzlibrary{positioning}
\usetikzlibrary{babel,arrows,automata}
%-------------------------------------

\usepackage[top=100pt,bottom=100pt,left=68pt,right=66pt]{geometry} % Geometria das folhas
%\usepackage[top=3cm, bottom=2cm, left=3cm, right=2cm]{geometry} % Geometria das folhas

\parskip 15pt					% distância entre parágrafos fixa
\parindent 30pt					%identação fixa

%----------------------
% Pacotes para a linha de código
%----------------------
\usepackage{xcolor}
% Definindo novas cores
\definecolor{verde}{rgb}{0,0.5,0}
% Configurando layout para mostrar codigos C++
\usepackage{listings}
\lstset{
  language=C,
  basicstyle=\ttfamily\small,
  keywordstyle=\color{blue},
  stringstyle=\color{verde},
  commentstyle=\color{red},
  extendedchars=true,
  showspaces=false,
  showstringspaces=false,
  numbers=left,
  numberstyle=\tiny,
  breaklines=true,
  backgroundcolor=\color{green!10},
  breakautoindent=true,
  captionpos=b,
  xleftmargin=0pt,
}
%----------------------
% Aumentar tamanho do título resumo
\makeatletter
\renewenvironment{abstract}{%
    \if@twocolumn
      \section*{\abstractname}%
    \else %% <- here I've removed \small
      \begin{center}%
        {\bfseries \Large\abstractname\vspace{\z@}}%  %% <- here I've added \Large
      \end{center}%
      \quotation
    \fi}
    {\if@twocolumn\else\endquotation\fi}
\makeatother
%----------------------


\begin{document}
    \onehalfspacing
\begin{titlepage}
	\begin{center}
	
	\begin{figure}[!ht]
	\centering
	\includegraphics[width=2cm]{./ufrn.jpg}
	\end{figure}
		Universidade Federal do Rio Grande do Norte \\ Centro de Tecnologia \\ Departamento de Engenharia Elétrica \\ ELE0648 - Tópicos Especiais em Transmissão de Energia Elétrica  - 2020.6 \\
		\vspace{15pt}
        \vspace{95pt}
        \textbf{\Large{Análise do Regime Permanente e Sobretensões Temporárias}}\\
		\vspace{3,5cm}
	\end{center}
	
	\begin{flushright}
			\item Levy Gabriel da Silva Galvão 
 	\end{flushright}
	\vspace{1cm}
	
	\begin{center}
		\vspace{\fill}
		Natal - RN, Outubro de 2020
	\end{center}
\end{titlepage}
%%%%%%%%%%%%%%%%%%%%%%%%%%%%%%%%%%%%%%%%%%%%%%%%%%%%%%%%%%%

% % % % % % % % %FOLHA DE ROSTO % % % % % % % % % %



\begin{titlepage}
	\begin{center}
	
	\begin{figure}[!ht]
	\centering
	\includegraphics[width=2cm]{./ufrn.jpg}
	\end{figure}

		Universidade Federal do Rio Grande do Norte \\ Centro de Tecnologia \\ Departamento de Engenharia Elétrica \\  ELE0648 - Tópicos Especiais em Transmissão de Energia Elétrica  - 2020.6 \\
\vspace{15pt}
        
        \vspace{85pt}
        
		\textbf{\Large{Análise do Regime Permanente e Sobretensões Temporárias}}\\
	%	\large{Modelo\\
     %   		Validação do modelo clássico}
			
	\end{center}
\vspace{1,5cm}
	
	\begin{flushright}

   \begin{list}{}{
      \setlength{\leftmargin}{4.5cm}
      \setlength{\rightmargin}{0cm}
      \setlength{\labelwidth}{0pt}
      \setlength{\labelsep}{\leftmargin}}

      \item Resumo referente ao assunto ministrado na disciplina de Tópicos Especiais em Transmissão de Energia Elétrica, como requisito para avaliação da primeira unidade.

      \begin{list}{}{
      \setlength{\leftmargin}{0cm}
      \setlength{\rightmargin}{0cm}
      \setlength{\labelwidth}{0pt}
      \setlength{\labelsep}{\leftmargin}}


            \item Orientador: Profº. Drº. José Tavares de Oliveira

      \end{list}
   \end{list}
\end{flushright}
\vspace{1cm}
\begin{center}
		\vspace{\fill}
		 Natal - RN, Outubro de 2020
			\end{center}
\end{titlepage}
\newpage
    \begin{abstract}

Regarding the rising importance of Orthogonal Frequency Division Multiplexing (OFDM) in recent communication systems, this work explains how an OFDM signal can be synthesized with the use of the Inverse Discrete Fourier Transform (IDFT) of which efficient implementation is the Inverse Fast Fourier Transform (IFFT).
\end{abstract}
    % Set Title, Author, and email
\title{ELE0646 - Chaves fusíveis}
\author{Levy Gabriel da S. G. \\ Engenharia elétrica - UFRN}

\maketitle
\thispagestyle{fancy}

\begin{itemize}
    \item Sobrecarga: não é considerado um defeito, com correntes de 10x a nominal;
    \item Curto-circuito: defeito, ocorrendo quando a impedância entre dois pontos é reduzida a valores próximos de zero, com intensidade maior que 10x a corrente nominal;
\end{itemize}

\textbf{Aplicações}

\begin{itemize}
    \item Proteção de TRAFOs de força de SEs de concessionária e by-pass de disjuntores/religadores de circuitos de distribuição de MT;
    \item Montagem nas estruturas de subestações;
    \item Constituídas por dois isoladores em uma base metálica;
\end{itemize}

\textbf{Partes constituintes}

\begin{itemize}
    \item Isolador;
    \item Suporte de fixação;
    \item Cartucho, porta-fusível ou "canela";
    \item Terminal de fonte;
    \item Terminal de carga;
    \item Articulação;
    \item Elo fusível (dentro do cartucho).
    \begin{itemize}
        \item Tipos:
        \begin{itemize}
            \item Botão;
            \item Argola ou olhal.
        \end{itemize}
        \item Partes:
        \begin{itemize}
            \item Elemento fusível;
            \item Tubinho;
            \item Rabicho.
        \end{itemize}
    \end{itemize}
\end{itemize}

\textbf{Curvas de atuação}

\begin{itemize}
    \item Curva dos tempos mínimos de fusão: curva mais baixa de cada elo (vermelho);
    \item Curva dos máximos tempos de interrupção: curva mais alta de cada elo (preto);
    \item O tempo de arco é a diferença de tempo entre as duas curvas;
\end{itemize}

\textbf{Tipos de ele fusível}

\begin{itemize}
    \item Elos fusíveis de distribuição:
    \begin{itemize}
        \item Elo tipo H (alto surto);
        \begin{itemize}
            \item Ação lenta, fabricado para pequenas correntes nominais, para proteção de TRAFOS de até 75kVA e bancos de capacitores;
            \item Elos da série H: 1H, 2H, 3H e 5H;
            \item Elos da série H não possuem capacidade de sobrecarga e começam a operar a partir de 1.5 vezes a sua corrente nominal num tempo de 300s.
        \end{itemize}
        \item Elo tipo K (elos rápidos);
        \item Elo tipo T (elos lentos).
        \begin{itemize}
            \item Os elos do tipo K e T são empregados na proteção de ramais de circuitos de distribuição de MT e proteção contra sobrecorrentes para TRAFOS entre 112.5kVA e 300kVA (acima desse valor a proteção deve ser realizada por disjuntores com relés);
            \item Para elos fusíveis de mesma corrente nominal, os elos da família K são mais rápidos do que os da família T, sendo assim mais utilizados;
            \item Classes da família K:
            \begin{itemize}
                \item Elos preferenciais: 6K, 10K, 15K, 25K, 40K, 65K, 100K, 140K e 200K;
                \item Elos não-preferenciais: 8K, 12K, 20K, 30K, 50K e 80K. 
            \end{itemize}
            \item Os elos K e T admitem sobrecargas de até 1.5 vezes as suas correntes nominais sem causar excesso de temperatura à chave-fusível;
            \item A fusão dos elos K e T ocorrem em aproximadamente 2 vezes os seus valores nominais em um tempo de 300 segundos.
        \end{itemize}
    \end{itemize}
   \item Elos fusíveis de força:
   \begin{itemize}
        \item Elo tipo EF (elos rápidos);
        \item Elo tipo ES (elos lentos).
    \end{itemize}
\end{itemize}

\textbf{Coordenação entre elos fusíveis}

\begin{itemize}
    \item O sistema será coordenado quando efeitos a jusante do elo fusível protetor não provocarem o desligamento do elo fusível protegido:
    \begin{itemize}
        \item Elemento protetor é aquele mais próximo ao defeito;
        \item Elemento protegido é a proteção de retaguarda, mais distante do ponto de defeito.
    \end{itemize}
    \item O elo protetor deve atuar primeiro que o protegido;
    \item Os elos fusíveis adjacentes das séries preferencial e não preferencial não apresentam coordenação (elo protetor 6K[preferencial] não coordenada com elo protegido de 8K[não preferencial], mas coordena com elo de 10K[preferencial] e 12K[não preferencial]);
    \item Geralmente utiliza-se a série preferencial por possuírem mais opções, assim aumentando flexibilidade.
\end{itemize}

\textbf{Especificação das chaves-fusíveis}

\begin{itemize}
    \item Tensão nominal;
    \item NBI;
    \item Corrente nominal;
    \item Capacidade de ruptura ou corrente de interrupção.
\end{itemize}

\textbf{Dimensionamento da chave-fusível}
\begin{itemize}
    \item A corrente do cartucho da chave-fusível deve ser igual ou superior à corrente admissível do fusível multiplicada por um fator K: $|I_{NOM}^{CF}| > K \times |I_{ADM}^{FS}|$
    \item A corrente de interrupção da chave-fusível deve ser igual ou superior ao valor assimétrico da corrente de curto-circuito no ponto de instalação da chave: $|I_{INT}^{CF}| \leq |I_{CC}^{ASM}|$
\end{itemize}

\textbf{Dimensionamento do elo-fusível}
\begin{itemize}
    \item Operar para curtos-circuitos no TRAFO e na rede secundária;
    \item Não confundir a sobrecarga permissível no TRAFO (para TRAFOS de distribuição admite-se sobrecarga de até $2 \times I_{NTRAFO}$ durante um tempo de 0.1s (6 ciclos de 60Hz));
    \item Não confundir a corrente de \textit{inrush}, estimada em $8 \text{ou} 12 \times I_{NTRAFO}$, durante um tempo de 0.1s (6 ciclos de 60Hz);
    \item Deve coordenar as proteções a montante;
    \item A curva de interrupção do elo deverá estar abaixo da curva térmica do TRAFO.
    \begin{table}[H]
\centering
\begin{tabular}{|c|c|c|}
\hline
Potência do transformador (kVA) & Corrente (A) & Fusível (A) \\ \hline
15                              & 0.63         & 1H          \\ \hline
30                              & 1.26         & 2H          \\ \hline
45                              & 1.88         & 3H          \\ \hline
75                              & 3.14         & 5H          \\ \hline
112.5                           & 4.71         & 6K          \\ \hline
150                             & 6.28         & 8K          \\ \hline
225                             & 9.41         & 10K         \\ \hline
300                             & 12.55        & 15K         \\ \hline
\end{tabular}
\end{table}
\end{itemize}


\begin{figure}[H]
\begin{center}
\includegraphics[width=16cm]{tabela1.PNG}  
\label{fig:1} 
\end{center}
\end{figure}

\begin{figure}[H]
\begin{center}
\includegraphics[width=16cm]{tabela2.PNG}  
\label{fig:2} 
\end{center}
\end{figure}

\begin{figure}[H]
\begin{center}
\includegraphics[width=16cm]{tabela3.PNG}  
\label{fig:3} 
\end{center}
\end{figure}

\begin{figure}[H]
\begin{center}
\includegraphics[width=16cm]{tabela4.PNG}  
\label{fig:4} 
\end{center}
\end{figure}
    \begin{thebibliography}{99}
\bibitem{b1} de Lauro Castrucci, Plínio Benedicto, e Anselmo Bittar. Controle automático. Grupo Gen-LTC, 2000.
\bibitem{b2} Kuo, Benjamin C., e Farid Golnaraghi. Automatic control systems. Vol. 9. Englewood Cliffs, NJ: Prentice-Hall, 1995.
\bibitem{b3} Introduction to Control Systems in Scilab. Disponível em: \url{http://www.openeering.com/}.
\bibitem{b4} Ogata, Katsuhiko. Modern control engineering. Upper Saddle River, NJ: Prentice Hall, 2009.
\bibitem{b5} Sedra, Adel S., et al. Microelectronic circuits. Oxford University Press, 2016.
\end{thebibliography}

\pagebreak
    \appendix

\section{Scilab \textit{scripts}}

\subsection{Ensaio da MI com rotor em vazio e bloqueado}

\begin{lstlisting}
// GRANDEZAS MEDIDAS COM ESTATOR EM Y

r_estator = [6.3, 6.3, 6.2]; 

// ensaio com rotor em vazio (u, v, w)

v_vazio = [377.9, 376, 382.5]/sqrt(3); 
i_vazio = [1.675, 1.537, 1.549];
p_vazio = [40, 22, 56];

// ensaio com rotor bloqueado

v_bloqueado = [64.5, 64.2, 65.1]/sqrt(3); 
i_bloqueado = [2.541, 2.513, 2.498];
p_bloqueado = [65, 64, 65];

// GRANDEZAS OBTIDAS DA PLACA

p_eixo_nominal = 1100;
n_eixo_nominal = 1715;
n_estator = 1800;
polos = 4;

// CÁLCULOS DOS PARÂMETROS

p_estator = r_estator.*i_vazio.^2;
p_rotacionais = p_vazio - p_estator;
x1_mais_xphi = sqrt((v_vazio./i_vazio).^2 - r_estator.^2);
r2 = (p_bloqueado./i_bloqueado.^2) - r_estator;
x1_mais_x2 = sqrt((v_bloqueado./i_bloqueado).^2 - (r_estator + r2).^2);
x1 = x1_mais_x2/2;
x2 = x1;
xphi = x1_mais_xphi - x1;
r1 = r_estator;

// CÁLCULO DE GRANDEZAS

Z1 = ((r1+x1*%i).*(xphi*%i))./(r1+(x1+xphi)*%i);
R1 = real(Z1);
X1 = imag(Z1);
Vth = (v_vazio.*xphi*%i)./(r1+(x1+xphi)*%i);

w_eixo_nominal = n_eixo_nominal*(2*%pi/60);
p_nominal = p_eixo_nominal + p_rotacionais;
t_eixo_nominal = p_nominal/w_eixo_nominal;
q1 = 3;

function func = f(x)
    temp1 = (1-x)/w_eixo_nominal;
    temp2 = q1*(abs(Vth)'.^2.*r2'./x);
    temp3 = ((R1'+r2'./x).^2+(X1'+x2').^2);
    func = ((temp1.*temp2)./(temp3))-t_eixo_nominal';
endfunction

sn1 = fsolve([0;0;0],f);
sn2 = fsolve([0.01;0.01;0.01],f);
sn3 = fsolve([1;1;1],f);

st_max = r2./sqrt(R1.^2+(X1+x2).^2);
w_sincrono = w_eixo_nominal./(1-sn2);
s = linspace(0.000001,1,1000);
for i = 1:3
    for j = 1:length(s)
        T(i,j) = (q1*(1/w_sincrono(i))*(abs(Vth(i))^2*r2(i)/s(j)))/((R1(i)+r2(i)/s(j))^2+(X1(i)+x2(i))^2);
    end
end

t_max = [max(T(1,:)),max(T(2,:)),max(T(3,:))];
s_max_idx = [find(T(1,:)==t_max(1)),find(T(2,:)==t_max(2)),find(T(3,:)==t_max(3))];
s_max = [s(s_max_idx(1)),s(s_max_idx(2)),s(s_max_idx(3))];
rc = sqrt(R1.^2+(X1+x2).^2)-r2;

for i = 1:3
    for j = 1:length(s)
        T_new(i,j) = (q1*(1/w_sincrono(i))*(abs(Vth(i))^2*(r2(i)+rc(i))/s(j)))/((R1(i)+(r2(i)+rc(i))/s(j))^2+(X1(i)+x2(i))^2);
    end
end

figure
plot2d(s,[T(1,:)',T(2,:)',T(3,:)'],[1,2,3]);
e=gce();

hl=captions(e.children,['Fase u';'Fase v';'Fase w']);
hl=captions(e.children,['Fase u';'Fase v';'Fase w'],'in_upper_right');

hl.legend_location='in_upper_right'
hl.fill_mode='on';

figure
plot2d(s,[T_new(1,:)',T_new(2,:)',T_new(3,:)'],[1,2,3]);
e=gce();

hl=captions(e.children,['Fase u';'Fase v';'Fase w']);
hl=captions(e.children,['Fase u';'Fase v';'Fase w'],'in_upper_right');

hl.legend_location='in_upper_right'
hl.fill_mode='on';
\end{lstlisting}


\pagebreak
\end{document}