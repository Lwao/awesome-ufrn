\begin{figure}[H]
\begin{center}
\tikz \node [scale=0.70, inner sep=0] {
\begin{tikzpicture} [ american, ]
    \draw (0,0) node[op amp] (opamp1) {}
    (opamp1.+) -- (-1.5,-0.5) node[ground]{}
    (opamp1.-) to[R, l_=$R_1$, -*] (-6,0.5) 
    (opamp1.out) 
    ;
    \draw (6,-0.5) node[op amp] (opamp2) {}
    (opamp2.+) -- (4.5,-1) node[ground]{}
    (opamp2.-)  to[R, l_=$R_3$]  (opamp1.out)
    (opamp2.out) to[short, -*] (8,-0.5)
    ;
    \draw (-6.5,0.5) node[left]{$V_{in}$}
    (8.5,-0.5) node[right]{$V_{out}$}
    (-1.5,0.5) to[short, *-] (-1.5,2) to[R, l=$R_2$] (1.5,2) to[short, -*] (1.5,0)
    (4.5,0) to[short, *-] (4.5,2) 
    (7.5,2) to[short, -*] (7.5,-0.5) 
    (4.5,2) to[R, l=$R_4$] (7.5,2)
    (-4.75,0.5) to[short, *-] (-4.75,2) to[C, l=$C_1$] (-2.5,2) to[short, -*] (-2.5,0.5)
    ;
    
\end{tikzpicture}
};
\end{center}
\caption{Circuito proposto para o PD.}
\label{ckt:pd} 
\end{figure}